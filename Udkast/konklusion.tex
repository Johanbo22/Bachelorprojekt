
% Konklusionen skal indeholde:
     % Svar på problemformuleringen
     % Svar på alle underspørgsmål i problemformuleringen
     % Svar på de metodiske begrænsninger 
 

%Hvordan kan GIS-modellen "Inundation Model" simulere stormflodshændelsen fra oktober 2023 og forudsige udbredelsen af fremtidige oversvømmelseshændelser?

%Til at besvare problemformuleringen stilles der følgende underspørgsmål:
%Hvor præcist kan Inundation Model simulere stormfloden der blev observeret den 20.-21. oktober 2023?
   % Til at besvare dette vil resultatet fra Inundation Model blive sammenlignet med kort over vandets udbredelse fra stormfloden i Aabenraa, Gedser Havn, Hesnæs og Præstø.
 % Hvilke arealanvendelser blev påvirket af oversvømmelserne under stormfloden?
 % Hvordan vil stormfloden se ud i slutningen af det nuværende århundrede som følge af et stigende havspejl?
 % Hvordan forventes en stormflod at påvirke studieområderne ved en statistisk 100-års hændelse ved et mellem og meget højt udledningsscenarie?
 % Diskutere hvorvidt Inundation Model kan være en brugbar del af kommunernes stormflodsanalyser og %planlægning af fremtidig kystsikring


På baggrund af de simulerede 2023-stormflodshændelser og fremtidige stormfloder af Aabenraa, Gedser, Hesnæs og Præstø konkluderes det at GIS-modellen Inundation Model har produceret  tilfredsstillende resultater. 
Modellen simulerede stormflodens udbredelse i Gedser Havn til 1,3 ha og Hesnæs til 0,2 ha mindre end den observerede stormflod. For de to andre områder, Aabenraa og Præstø, var der eksterne faktorer, henholdsvis beredskabstiltag og en ødelagt sluse, der gjorde at modellen over- og underestimerede oversvømmelserne i forhold til den observerede stormflod. Dette gjorde at det ikke har været muligt at korrekt bedømme modellens resultat i forhold til observeret data for de to områder.\\
Kvantificeringen af påvirkede arealanvendelser fremhæver at det primært har været naturområder, bebyggede områder, infrastruktur og rekreative arealer der blev oversvømmet af stormfloden i 2023, mens arealanvendelser som erhverv og landbrug i mindre grad blev påvirket. Derudover har modellen vist at være fornuftig til at simulere fremtidige stormflodshændelser og give et bud på hvordan udstrækningen af fremtidige oversvømmelserne vil se ud, som gør det muligt for både kommuner og private personer at hurtigt få et overblik og muligheden for at planlægge forebyggelsen af potentielle skader forårsaget af en lignende stormflod i fremtiden.\\

Selvom modellens tilfredsstillende resultater for to af områderne, har det derimod ikke været muligt at afgøre med sikkerhed hvorvidt modellen vil over- eller underestimere oversvømmelsens udstrækning under en stormflod. Hvis en fremtidig undersøgelse laves, anbefales det at der anvendes en større samling af studieområder til at yderligere teste modellen på, der kan bidrage til et mere afgørende bud på om modellen vil over- eller underestimere når den skal simulere fremtidige stormflodshændelser. Kombineret med modellens høje krav til kvaliteten af DTM, opdateret og omfattende hydrologiske tilpasninger og at være eksklusiv til den proprietære software ArcGIS Pro, reduceres modellens potentielle anvendelighed indenfor stormflodsanalyse ikke kun i Danmark, men også i Europa og resten af verden.\\

På trods af dette vurderes Inundation Modellen til at være af høj kvalitet. Hvis modellen bringes til et ikke-kommercielt GIS-produkt vil modellen kunne sammen med lav processeringstid på god hardware, høj brugervenglighed og muligheden for at kombinere med andre rumlige dataprodukter, lægge op til et solidt grundlag for et stærkt og fleksibelt redskab til brug i den indledende fase af stormflods- og kystsikrings planlægningen fremover.
 

