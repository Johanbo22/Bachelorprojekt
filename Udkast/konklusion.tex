
% Konklusionen skal indeholde:
     % Svar på problemformuleringen
     % Svar på alle underspørgsmål i problemformuleringen
     % Svar på de metodiske begrænsninger 
     % Svar på 

%Hvordan kan GIS-modellen "Inundation Model" simulere stormflodshændelsen fra oktober 2023 og forudsige udbredelsen af fremtidige oversvømmelseshændelser?

%Til at besvare problemformuleringen stilles der følgende underspørgsmål:
%Hvor præcist kan Inundation Model simulere stormfloden der blev observeret den 20.-21. oktober 2023?
   % Til at besvare dette vil resultatet fra Inundation Model blive sammenlignet med kort over vandets udbredelse fra stormfloden i Aabenraa, Gedser Havn, Hesnæs og Præstø.
 % Hvilke arealanvendelser blev påvirket af oversvømmelserne under stormfloden?
 % Hvordan vil stormfloden se ud i slutningen af det nuværende århundrede som følge af et stigende havspejl?
 % Hvordan forventes en stormflod at påvirke studieområderne ved en statistisk 100-års hændelse ved et mellem og meget højt udledningsscenarie?
 % Diskutere hvorvidt Inundation Model kan være en brugbar del af kommunernes stormflodsanalyser og %planlægning af fremtidig kystsikring

Antallet og intensiteten af stormfloder vil i fremtiden stige i takt med et stigende havspejl. Derfor er det nødvendigt at der udvikles hurtige, præcise og tilgængelige modeller der kan simulere stormfloders indvirkning på kystbyer og samfundet.\\
Inundation Modellen har simuleret stormflodshændelsen fra 2023 med fornuftige resultater. Modellen formåede at simulere vandets udbredeslse i Gedser Havn og Hesnæs i det samme omfang, som observeret under stormfloden. Aabenraa og Præstø blev derimod henholdsvis over- og underrepræsenteret grundet eksterne faktorer, som beredskabstiltag og en ødelagt sluseport.  


