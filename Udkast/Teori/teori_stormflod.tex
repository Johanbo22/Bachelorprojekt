
\subsection{Stormfloder} \label{Afsnit: Stormfloder}

En stormflod er betegnelsen for usædvaneligt højvande i relation til kraftig vind \citep{shoreline_management_guidelines}.
Styrken og påvirkningen af en stormflod på en lokalitet er afhængig af en række faktorer herunder orienteringen af kysten i forhold til stormen, hvor kraftig stormen er og de lokale havbunds- og dybdeforhold \citep{noaa_storm, shoreline_management_guidelines}.\\

I Danmark er det Vadehavskysten der ofte er det mest udsatte område, da vinden i Danmark primært kommer fra vest \citep{cappelen_dmi_2020}. Over længere perioder med vesten vind, vil vandet i Nordsøen blive presset ind Kattegat og længere ned i de indre danske farvande. Hvis vinden er langvarig, vil vandet over tid blive presset igennem bælterne ved Lillebælt, Storebælt og Øresund og videre ind i Østersøen og nord imod den Botniske Bugt. \citep{kystdirektoratet_stormfloder}. Dette fænomen kaldes for \textit{"preconditioning"} og beskriver hvordan vandstanden stiger i Østersøen inden begyndelsen af en storm \citep{kiesel_brief_2024, weisse_sea_2021}. \\   

Når vinden derefter aftager eller ændrer retning vil alt vandet der er blevet presset ind i Østersøen, skvulpe tilbage mod bælterne i Danmark. Dette kaldes for \textit{"badekarseffekten"} og er illustreret i figur \ref{Figur: Bathtub effect} \citep{kystdirektoratet_stormfloder, egusphere_baltic}. Bælterne i de indre danske farvande vil herefter fungere som flaskehalse for vandmasserne der skvulper tilbage fra Østersøen og resultere i oversvømmelse i de indre danske farvande \citep{egusphere_baltic}.
\begin{figure}[H]
    \centering
    \includegraphics[width=0.8\linewidth]{images/teori/bathtub effect graphics.jpg}
    \caption{Illustration af "badekarseffekten". De sorte pile indikerer vindretning og de blå og røde pile indikerer bevægelsen af vandmassen. Kilde: Egen illustration, baggrundskort fra Esri}
    \label{Figur: Bathtub effect}
\end{figure}
Nævneværdige stormfloder som stormfloden den 1.-2. november 2006, den 20.-21. oktober 2023 og den 12.-14. november 1872 blev forårsaget af kraftig østenvind og \textit{"badekarseffekten"} \hspace{1.5cm} \citep{kystdirektoratet_stormfloder}.

\subsection{Stormfloden den 20 oktober 2023} \label{Afsnit: Stormfloden den 20 oktober 2023}
I første halvdel af oktober 2023 blev der observeret moderate vindforhold med gennemsnitlig middelvind på 5,5 m/s og en gennemsnitlig maksimal 10.min middelvind på 18,3 m/s fra vest, hvilket medførte en betydelig vandtransport ind gennem Kattegat og videre i Østersøen \citep{dmi_vejrarkiv}. \\
Den 18. oktober skiftede vindretningen til øst (figur \ref{Figur: Vinddata Danmark}) grundet trykforskelle mellem et højtryksystem over Skandinavien og et lavtrykssystem over Storbritannien \citep{kiesel_brief_2024}, og middelvindhastigheden steg i hele landet til 12,2 m/s og maksimale 10.min middelvind til 28,3 m/s om aftenen den 20. oktober 2023. 
\begin{figure} [H]
    \centering
    \includegraphics[width=0.8\linewidth]{images/vinddata_grafer/Danmark_vinddata.pdf}
    \caption{Vinddata for Danmark i oktober 2023 der viser middelvind, højeste 10.min middelvind og højeste vindhastighed i m/s. De sorte pile i bunden indikerer vindretningen. Kilde: Data fra \cite{dmi_vejrarkiv} }
    \label{Figur: Vinddata Danmark}
\end{figure}

Kombinationen af kraftig vind fra øst og stor vandtransport ind i Østersøen over en periode på 15 dage resulterede ifølge \cite{kystdirektoratet_stormflod2023} i markante vandstandsstigninger i de indre farvande af Syddanmark den 20. oktober. Den maksimale vandstand målt under stormfloden blev målt i Aabenraa Havn på 2,16 m over dagligt vande \citep{damberg_vaerste_2023}. Disse forhold forårsagede omfattende skader af sommerhusområder, kystnære områder og byer \citep{kystdirektoratet_stormflod2023, naturskaderadet_anmeldelser_2023}




