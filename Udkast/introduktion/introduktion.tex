
\subsection{Hvad er viden på området} % skal fjernes (unødvendig)

% hvorfor er stormfloder et problem
Stormfloder er i blandt de farligste naturlige farer for kystnære egne. Områder med høj befolkningstæthed og vigtigheden af kystnære byer til havne og infrastruktur gør disse områder sårbare over for oversvømmelser, skader på ejendomme og tab af af menneskeliv \citep{kaniewski_solar_2016}. Stormfloder har høj økonomiske og sundhedsmæssige konsekvenser og forventes at stige til 342 mia. kr. om året globalt frem mod 2060 eftersom en stigning af mennesker og udvikling langs kystnære områder forventes \citep{kaniewski_solar_2016,hallegatte_future_2013}.\\



% hvad ved vi om stormflodsmodellering
% artiker:  Poulter and Halpin (2008), Li et al (2014), Leijnse et al (2021), Delamater et al (2012), Dedekorkut-Howes et al (2020), Al-Attabi et al (2023), Balstrøm og Kirby (2022)

% hvordan er stormflodsmodelleringen igennem tiden
% Haight et al (2017), 
% hvad er status quo på stormflodsproblemet (eksempler på stormflodsinitiativer lige nu i Danmark) og modellering af det lige nu

\subsection{Føre frem til problemformuleringen}

Dette fører frem til dette projekts hovedproblemstilling 

\subsection{Problemformulering, underspørgsmål}
\begin{center}
    \fbox{\parbox{\textwidth}{\centering Hvordan kan GIS-modellen \textit{"Inundation Model"} simulere stormflodshændelsen fra oktober 2023 og forudsige udbredelsen af fremtidge oversvømmelseshændelser?}}
\end{center}

Til at besvare denne problemformulering stilles der følgende underspørgsmål:
\begin{itemize}
    \item Hvor præcist kan Inundation Model simulere stormfloden der blev observeret den 20.-21. oktober 2023?
    \begin{itemize}
        \item Til at besvare dette vil resultatet fra Inundation Model blive sammenlignet med kort over vandets udbredelse fra stormfloden i Aabenraa, Gedser Havn, Hesnæs og Præstø.
    \end{itemize}
    \item Hvilke arealanvendelser blev påvirket af oversvømmelserne under stormfloden?
    \item Hvordan vil stormfloden se ud i slutningen af det nuværende århundrede som følge af et stigende havspejl?
    \item Hvordan forventes en stormflod at påvirke studieområderne ved en statistisk 100-års hændelse ved et mellem og meget højt udledningsscenarie?
    \item Diskutere hvorvidt Inundation Model kan være en brugbar del af kommunernes stormflodsanalyser og planlægning af fremtidig kystsikring
\end{itemize}