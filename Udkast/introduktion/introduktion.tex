
% hvorfor er stormfloder et problem
Stormfloder udgør blandt de mest alvorlige naturfarer for kystnære områder. Høj befolkningstæthed samt den strategiske betydning af kystbyer for havn og infrastruktur medfører, at disse områder er særlig sårbare over for oversvømmelser, betydelige materielle skader og tab af menneskeliv \citep{kaniewski_solar_2016}. Stormfloder har omfattende økonomiske og sundhedsmæssige konsekvenser og kan forårsage afbrydelser i transport, beskadigelse af energi- og infrastruktursystemer samt langvarige strømafbrydelser, hvilke yderligere kan belaste beredskabs- og sundhedssektoren \citep{lane_health_2013}. Det forventes at de økonomiske konsekvenser af stormfloder globalt vil stige til 413 mia. kr. årligt frem mod 2050, ligesom udgifterne til etablering af nye beskyttende diger forventes at vokse som følge af øget befolkningstal og byudvikling langs kysterne \citep{prahl_damage_2018,hallegatte_future_2013}.\\
I takt med det stigende globale havnniveau, drevet af klimaforandringer, forventes en øget intensitet og hyppighed af stormfloder, hvilket yderligere vil forstærke presset på infrastruktur, beredskab og sundhedssystemer og nødvendigheden af adaptiv og proaktiv kystsikring \citep{marsooli_climate_2019, dedekorkut-howes_when_2020}. \\

% hvad ved vi om stormflodsmodellering
Evnen til at forudsige og tilpasse kystsamfund til fremtidens stormfloder er derfor afgørende. Dette nødvendiggør udvikling og anvendelse af avancerede modelleringsværktøjer. De tidligste stormflodsmodeller var baseret på empiriske målinger af trykforskelle under storme og førte til de første computerdrevne modeller, der anvendte empiriske formler og lokale geografiske korrektioner \citep{massey_history_2007}. Senere blev modeller udviklet og testet ved at rekonstruere historiske stormfloder. Dette blev en tilgang brugt til at skabe omfattende databaser med historisk data af stormfloder til at træne fremtidige stormflodsmodeller og udføre trendanalyse til forudsige ændringer i stormflodsmønstre \citep{tadesse_database_2021, dang_dataset_2024}. Mod slutningen af det 20. århundrede blev modellerne gradvist mere avanceret med inddragelse af flere parametre til at beskrive lokale forhold. Samtidig muliggjorde teknologisk fremskridt øget computerkapacitet udviklingen af kraftige hydrodynamiske modeller, der kan forudsige stormfloder på stor skala med kort modelleringstid \citep{tadesse_database_2021, massey_history_2007}.\\
I dag anvendes avancerede modeller, der integrerer lokale og regionale klimatologiske parametre såsom vindhastighed, vindretning, kystmorfologi, fristræk, batrymetri og geologi, hvilket muliggør realtidsmodellering af stormfloder. Modeller som MIKE21/3 hydrodynamiske modeller \citep{dhi_mike_2024} og \cite{adcirc_introduction_nodate} kan simulere komplekse hydrodynamiske forhold i både 2D og 3D, herunder sedimenttransport samt kyst- og bunderosion forårsaget af stormfloder. SFINCS-modellen, først beskrevet af \cite{leijnse_modeling_2021} kan simulere oversvømmelser forårsaget af både stormfloder og ekstreme nedbørshændelser samtidig, som det der blev observeret under orkanen Katrina i 2005 \citep{reible_hurricane_2007}. Da størstedelen af disse modeller er komplekse og opererer på regionale eller nationale skala bliver der ofte anvendt lavtopløsligt data for at muliggøre realtidssimuleringer som respons til kommende stormflodshænelser.\\

Det er således ligeså væsentligt at udvikle modeller, der hurtigt og tilgængeligt kan give et overblik over potentielle stormflodsscenarier og deres påvirkning på både regionale og lokale niveauer \citep{balstrom_kirby_inundation}. Rasterbaserede oversvømmelsesmodeller indenfor geografiske informationssystemer (GIS) er en effektiv metode til hurtig visualisering og analyse af stormflodsoversvømmelser. Metoder såsom bathtub-metoden, der anvender en simpel segmentering af højdeværdier i en højdemodel ved et givent vandstandsniveau \citep{poulter_raster_2008}, samt cost-distance algoritmer, der sikrer at vandets logiske bevægelse fra en punktkilde gennem terrænet og samtidig tager højde for naturlige barrierer i terrænet \citep{li_delineating_2014}, er veldokumenteret og udbredte.\\
Rasterbaserede stormflodsmodeller, der bygger på disse principper, kan dermed øge tilgængeligheden af semi-avancerede analyseværktøjer til indledende stormflodsvurderinger. De drager fordel af forbedret datapopløsning og teknologiske fremskridt inden for computerkraft, samtidig med at de reducerer modellernes kompleksitet til minimum \citep{balstrom_kirby_inundation, li_delineating_2014}. En af de modeller der bygger på dette princip er Inundation Modellen fra \cite{balstrom_kirby_inundation}.\\

Siden 1991 frem mod 2017 er der i Danmark blevet registreret i alt 27 stormfloder, hvilket svarer til en gennemsnitlig forekomst på knap én stormflod pr. år. Prognoser indikerer, at denne hyppighed kan stige op til to stormfloder årligt ved udgangen af det 21. århundrede \citep{erhvervsministeriet_fremtidige_nodate}. Den senest officielt anekendte stormflod indtraf i midten af oktober 2023. For at styrke danske kommuners og kystnære samfunds beredskab over for fremtidige stormflodshændelser er det derfor essentielt at foretage en kritisk vurdering af de metoder og modeller, der anvendes i stormflodsmodellering. Dette indebærer at undersøge modellernes evne til at korrekte replikere tidligere observerede stormflodshændelser, hvilket tjener som en validering af modellens evne til at simulere fremtidige stormfloders omfang.\\

Det er med dette udgangspunkt at der undersøges følgende som projektets centrale problemstilling. 
\begin{center}
    \fbox{\parbox{\textwidth}{\centering Hvordan kan GIS-modellen \textit{"Inundation Model"} simulere stormflodshændelsen fra oktober 2023 og forudsige udbredelsen af fremtidige oversvømmelseshændelser?}}
\end{center}

Til at besvare problemformuleringen stilles der følgende underspørgsmål:
\begin{itemize}
    \setlength{\itemsep}{0pt}
    \setlength{\parskip}{0pt}
    \setlength{\parsep}{0pt}
    \item Hvor præcist kan Inundation Model simulere stormfloden der blev observeret den 20.-21. oktober 2023?
    \begin{itemize}
        \item Til at besvare dette vil resultatet fra Inundation Model blive sammenlignet med kort over vandets udbredelse fra stormfloden i Aabenraa, Gedser Havn, Hesnæs og Præstø.
    \end{itemize}
    \item Hvilke arealanvendelser blev påvirket af oversvømmelserne under stormfloden?
    \item Hvordan vil stormfloden se ud i slutningen af det nuværende århundrede som følge af et stigende havspejl?
    \item Hvordan forventes en stormflod at påvirke studieområderne ved en statistisk 100-års hændelse ved et mellem og meget højt udledningsscenarie?
    \item Diskutere hvorvidt Inundation Model kan være en brugbar del af kommunernes stormflodsanalyser og planlægning af fremtidig kystsikring
\end{itemize}