% VIGTIGT! For at dokumentet kan compile, skal man benytte LuaLaTeX som compiler. Dette kan gøres i Overleaf ved at gå til Menu -> Settings -> Compiler -> Vælg LuaLaTeX
\documentclass[12pt]{article}
\usepackage{KUstyle}
\usepackage[margin=0.8in]{geometry} % Pakke til sidemargin
\usepackage{setspace} % Pakke til halvanden linjeafstand
\usepackage{tabularx} % Pakke til tabel
\usepackage{fontspec}
\usepackage{mathtools} % til at skrive matematiske symboler
\usepackage{amsmath} % bruges til ligninger og matematik
\usepackage{amssymb} % symboler og andre ting og sager
\usepackage{graphicx} % inkluderer grafik og specielle grafiske modulatorer
\usepackage{float} % inkluderer float som en konfiguerations enhed
\graphicspath{{./images/}} % stien til billeder
\usepackage{natbib} % bibliografi.bst style config
\bibliographystyle{jmr} % bibliografi stilen. denne stil laver (navn, dato). Dokumentation findes i jmr.bst
\usepackage[danish]{babel} % en pakke til dansk, så labels er på dansk
\usepackage{xltxtra} % til at skrive sub- og superscripts
\usepackage{times}
\usepackage{gensymb} % symbolpakke
\usepackage{hyperref} % referencer
\usepackage{array} % til tabeller
\usepackage[table, xcdraw]{xcolor} % til tabeller
\usepackage{longtable} % til tabeller
\usepackage{caption} % til captions
\usepackage{multirow} % til tabeller 
\usepackage{subcaption} % til tabeller
\usepackage{colortbl} % til tabeller
\usepackage{makecell} % til tabeller
\usepackage{booktabs} % til tabeller
\usepackage[normalem]{ulem} %no one knows what this does.
\useunder{\uline}{\ul}{} % øhh noget med namespace definition, tror jeg
\usepackage{xurl} %ikke brugt
\usepackage{wrapfig} %N/A
\usepackage{subcaption}
\usepackage{minted} % til kode
\usepackage{tabularray} %tabel
\UseTblrLibrary{diagbox}
\usepackage{booktabs,siunitx,array,threeparttable} %%tabel
\usepackage{diagbox} % dialog kasse, 
\usepackage{setspace}%distance for marginkontrol



\setlength{\bibhang}{2em}



\hypersetup{breaklinks, colorlinks, urlcolor=black
}

\captionsetup[figure]{font=small,labelfont=bf, labelsep=period}
\captionsetup[figure]{justification=l, singlelinecheck=false}
\captionsetup[subfigure]{font=footnotesize, labelfont=bf, justification=centering, position=top, aboveskip=2pt}
\captionsetup[table]{font=small, labelfont=bf, labelsep=period}
\captionsetup[table]{justification=l,singlelinecheck=false}





\tolerance=1
\emergencystretch=\maxdimen
\hyphenpenalty=10000
\hbadness=10000

\setmainfont{Times New Roman}
\setlength{\parindent}{0pt}



% Nedenstående ændrer indholdet på forsiden
\ptype{Bachelorprojekt i geografi og geoinformatik}
\author{Johan Bo Kjær}
\title{Stormfloder i det 21. århundrede}
\subtitle{En kvalitetsundersøgelse af Inundation Model}
\date{Dato: {\today}}
\advisor{Vejleder: Mikkel Fruergaard}
\fpimage{bil.jpg} % Hvis intet billede ønskes, skal kommandoen fjernes
\fpcredit{Aabenraa midtby under stormfloden i oktober 2023. Kredit: Meks I/S}


\begin{document}


\maketitle


\onehalfspacing


\begin{table}[h]
\def\arraystretch{1.5}
\begin{tabularx}{\textwidth}{l X}
Institutnavn: & {Institut for Geovidenskab og Naturforvaltning}  \\
Forfatter: & {Johan Bo Kjær} \\
KU-brugerID: & {MCG811} \\
Titel: & {Stormfloder i det 21. århundrede, En kvalitetsundersøgelse af Inundation Model} \\
Emnebeskrivelse: & Dette projekt ønsker at undersøge kvaliteten og præcisionen af stormflodsmodellen Inundation Model ved at simulere stormflodshændelsen fra oktober 2023 på fire havnebyer i Danmark. Til at komplimentere denne analysere blev der undersøgt hvilke arealanvendelser der blev påvirket under stormfloden samt hvordan en fremskrevet og en statistisk stormflod vil påvirke områderne.\\
Nøgleord: & GIS-modellering, klimaforandringer, kyst, oversvømmelse, stormfloder \\
Vejleder: & Mikkel Fruergaard \\
Dato: & \today \\
Afleveringsdato: & 4. juni 2025 kl. 12:00 \\
Antal anslag: &  \\
Totale antal sider: &  \\
Tak til: & Aabenraa, Guldborgsund og Vordingborg kommune for adgang til data omkring oversvømmelsen fra stormfloden og det gode samarbejde\\
& Aabenraa Havn og COWI for indsigt i data og rapporter. \\
Note: & Projektet i skrevet i LaTex med den seneste version af TexLive og bruger LuaLaTex som compiler. Grafer og dataprocessering er håndteret i Python.\\
& Alt source-koden for Tex og python filer kan findes på min GitHub-repository "https://github.com/Johanbo22/Bachelorprojekt". Billederne fra Meks I/S er ikke tilgængelige pga. ophavsret. 
\end{tabularx} 
\end{table}

% begin: renewcommands:
\renewcommand{\contentsname}{Indholdsfortegnelse} % ændrer navnet på indholdsfortegnelsen
\renewcommand{\abstractname}{Abstract} %nanvet på abstract
\renewcommand{\refname}{} % sørger for at der ikke er en ekstra titel i referencelisten.

% end: renewcommands:

\newpage
% her er en begyndelse det engelske del af resumeét
\begin{table}[h]
\def\arraystretch{1.5}
\begin{tabularx}{\textwidth}{l X}
Project title: & Coastal Storm Surges in the 21st century \\ & A quality assessment of the Inundation Model \\
Author: & {Johan Bo Kjær} \\
Keywords: & \textit{assessment, climate change, GIS-modelling, inundation, storm surge}
\end{tabularx}
\end{table}
\begin{abstract} % 1000 anslag

    Coastal storm surges portend a major danger for coastal societies around the world due to the increasing sea level and the population growth along coastlines. As such fast and precise modelling of storm surges' extent on coastal towns are vital in planning and mangement.\\
    
    The focus of this study has therefore been to examine the raster-based storm surge model Inundation Model and assess its ability to simulate the storm surge of 2023 in the Danish coastal towns of Aabenraa, Gedser, Hesnæs and Præstø.\\
    
    This was an spatial and quantitative assessment of the models result compared with maps from the four study towns of the extent of the flooding occured during the storm surge in 2023. To compliment this, an analysis of the affected land use as well as a calculation and a simulation of a projected and a statistical storm surge based on two SSP scenarios were conducted.\\
    
    Gedser and Hesnæs were modelled to slightly smaller inundation extents as the observed event, while Aabenraa had a larger inundatation simulated due to emergency measures taken during the storm and Præstø had a smaller inundated extent due to an unforseen burst of a waterlock. The main land use affected were built-up and nature areas and the projected storm surges show large parts of the towns inundated due to sea level rise.\\
    
    The Inundation Model showed that is a relatively good model for fast and relative precise modelling of the 2023 storm surge event with two sites matching observed events. However due to the model being exclusive to a proprietary software and requiring high resolution datasets it's usage in municipalities for storm surge mangement is limited.

    
\end{abstract}

\newpage

\tableofcontents
\newpage


\section{Introduktion}

% hvorfor er stormfloder et problem
Stormfloder udgør blandt de mest alvorlige naturfarer for kystnære områder. Høj befolkningstæthed samt den strategiske betydning af kystbyer for havn og infrastruktur medfører, at disse områder er særlig sårbare over for oversvømmelser, betydelige materielle skader og tab af menneskeliv \citep{kaniewski_solar_2016}. Stormfloder har omfattende økonomiske og sundhedsmæssige konsekvenser og kan forårsage afbrydelser i transport, beskadigelse af energi- og infrastruktursystemer samt langvarige strømafbrydelser, hvilke yderligere kan belaste beredskabs- og sundhedssektoren \citep{lane_health_2013}. Det forventes at de økonomiske konsekvenser af stormfloder globalt vil stige til 413 mia. kr. årligt frem mod 2050, ligesom udgifterne til etablering af nye beskyttende diger forventes at vokse som følge af øget befolkningstal og byudvikling langs kysterne \citep{prahl_damage_2018,hallegatte_future_2013}.\\
I takt med det stigende globale havnniveau, drevet af klimaforandringer, forventes en øget intensitet og hyppighed af stormfloder, hvilket yderligere vil forstærke presset på infrastruktur, beredskab og sundhedssystemer og nødvendigheden af adaptiv og proaktiv kystsikring \citep{marsooli_climate_2019, dedekorkut-howes_when_2020}. \\

% hvad ved vi om stormflodsmodellering
Evnen til at forudsige og tilpasse kystsamfund til fremtidens stormfloder er derfor afgørende. Dette nødvendiggør udvikling og anvendelse af avancerede modelleringsværktøjer. De tidligste stormflodsmodeller var baseret på empiriske målinger af trykforskelle under storme og førte til de første computerdrevne modeller, der anvendte empiriske formler og lokale geografiske korrektioner \citep{massey_history_2007}. Senere blev modeller udviklet og testet ved at rekonstruere historiske stormfloder. Dette blev en tilgang brugt til at skabe omfattende databaser med historisk data af stormfloder til at træne fremtidige stormflodsmodeller og udføre trendanalyse til forudsige ændringer i stormflodsmønstre \citep{tadesse_database_2021, dang_dataset_2024}. Mod slutningen af det 20. århundrede blev modellerne gradvist mere avanceret med inddragelse af flere parametre til at beskrive lokale forhold. Samtidig muliggjorde teknologisk fremskridt øget computerkapacitet udviklingen af kraftige hydrodynamiske modeller, der kan forudsige stormfloder på stor skala med kort modelleringstid \citep{tadesse_database_2021, massey_history_2007}.\\
I dag anvendes avancerede modeller, der integrerer lokale og regionale klimatologiske parametre såsom vindhastighed, vindretning, kystmorfologi, fristræk, batrymetri og geologi, hvilket muliggør realtidsmodellering af stormfloder. Modeller som MIKE21/3 hydrodynamiske modeller \citep{dhi_mike_2024} og \cite{adcirc_introduction_nodate} kan simulere komplekse hydrodynamiske forhold i både 2D og 3D, herunder sedimenttransport samt kyst- og bunderosion forårsaget af stormfloder. SFINCS-modellen, først beskrevet af \cite{leijnse_modeling_2021} kan simulere oversvømmelser forårsaget af både stormfloder og ekstreme nedbørshændelser samtidig, som det der blev observeret under orkanen Katrina i 2005 \citep{reible_hurricane_2007}. Da størstedelen af disse modeller er komplekse og opererer på regionale eller nationale skala bliver der ofte anvendt lavtopløsligt data for at muliggøre realtidssimuleringer som respons til kommende stormflodshænelser.\\

Det er således ligeså væsentligt at udvikle modeller, der hurtigt og tilgængeligt kan give et overblik over potentielle stormflodsscenarier og deres påvirkning på både regionale og lokale niveauer \citep{balstrom_kirby_inundation}. Rasterbaserede oversvømmelsesmodeller indenfor geografiske informationssystemer (GIS) er en effektiv metode til hurtig visualisering og analyse af stormflodsoversvømmelser. Metoder såsom bathtub-metoden, der anvender en simpel segmentering af højdeværdier i en højdemodel ved et givent vandstandsniveau \citep{poulter_raster_2008}, samt cost-distance algoritmer, der sikrer at vandets logiske bevægelse fra en punktkilde gennem terrænet og samtidig tager højde for naturlige barrierer i terrænet \citep{li_delineating_2014}, er veldokumenteret og udbredte.\\
Rasterbaserede stormflodsmodeller, der bygger på disse principper, kan dermed øge tilgængeligheden af semi-avancerede analyseværktøjer til indledende stormflodsvurderinger. De drager fordel af forbedret datapopløsning og teknologiske fremskridt inden for computerkraft, samtidig med at de reducerer modellernes kompleksitet til minimum \citep{balstrom_kirby_inundation, li_delineating_2014}. En af de modeller der bygger på dette princip er Inundation Modellen fra \cite{balstrom_kirby_inundation}.\\

Siden 1991 frem mod 2017 er der i Danmark blevet registreret i alt 27 stormfloder, hvilket svarer til en gennemsnitlig forekomst på knap én stormflod pr. år. Prognoser indikerer, at denne hyppighed kan stige op til to stormfloder årligt ved udgangen af det 21. århundrede \citep{erhvervsministeriet_fremtidige_nodate}. Den senest officielt anekendte stormflod indtraf i midten af oktober 2023. For at styrke danske kommuners og kystnære samfunds beredskab over for fremtidige stormflodshændelser er det derfor essentielt at foretage en kritisk vurdering af de metoder og modeller, der anvendes i stormflodsmodellering. Dette indebærer at undersøge modellernes evne til at korrekte replikere tidligere observerede stormflodshændelser, hvilket tjener som en validering af modellens evne til at simulere fremtidige stormfloders omfang.\\

Det er med dette udgangspunkt at der undersøges følgende som projektets centrale problemstilling. 
\begin{center}
    \fbox{\parbox{\textwidth}{\centering Hvordan kan GIS-modellen \textit{"Inundation Model"} simulere stormflodshændelsen fra oktober 2023 og forudsige udbredelsen af fremtidige oversvømmelseshændelser?}}
\end{center}

Til at besvare problemformuleringen stilles der følgende underspørgsmål:
\begin{itemize}
    \setlength{\itemsep}{0pt}
    \setlength{\parskip}{0pt}
    \setlength{\parsep}{0pt}
    \item Hvor præcist kan Inundation Model simulere stormfloden der blev observeret den 20.-21. oktober 2023?
    \begin{itemize}
        \item Til at besvare dette vil resultatet fra Inundation Model blive sammenlignet med kort over vandets udbredelse fra stormfloden i Aabenraa, Gedser Havn, Hesnæs og Præstø.
    \end{itemize}
    \item Hvilke arealanvendelser blev påvirket af oversvømmelserne under stormfloden?
    \item Hvordan vil stormfloden se ud i slutningen af det nuværende århundrede som følge af et stigende havspejl?
    \item Hvordan forventes en stormflod at påvirke studieområderne ved en statistisk 100-års hændelse ved et mellem og meget højt udledningsscenarie?
    \item Diskutere hvorvidt Inundation Model kan være en brugbar del af kommunernes stormflodsanalyser og planlægning af fremtidig kystsikring
\end{itemize}


\section{Teori}


\subsection{Raster datamodel} \label{Afsnit: Raster data model}

En rastermodel er en datamodel der er bygget op af et regulært netværk af celler organiseret i et gitterformat \citep{bolstad_gis_2022, esri_raster}. Hver celle i en raster model er karakteriseret af en celle dimension, som definerer cellestørrelsen ud fra cellens længde i X- og Y-retningen \citep{bolstad_gis_2022}. Cellestørrelsen repræsenterer således den rumlige opløsning af raster datamodellen og fungerer som en indikator for den rumlige præcision, da hver celles koordinat er givet ud fra centrum af cellen. En større cellestørrelse vil derfor resultere i en højere rumlig usikkerhed, mens en mindre cellestørrelse vil medføre en lavere rumlig usikkerhed. En større cellestørrelse medfører også et større datasæt og optager mere plads i databaser \citep{bolstad_gis_2022}. Rastermodellen er derfor en afvejning af opløsning og filstørrelse.\\

Hver cele i en raster indeholder en værdi, der repræsenterer information om det pågældende geografiske område. Disse værdier kan enten være numeriske eller kategoriske, afhængigt af den type data, der ønskes repræsenteret. Numeriske værdier anvendes typisk til at beskrive kontinuerlige data, hvor værdierne kan variere gradvist fra celle til celle. Et eksempel på dette er højdemodeller, hvor hver celle indeholder en talværdi, der angiver terrænets højde (figur \ref{Subfig: Kontinuer raster}). Andre eksempler på kontinuerlige rasterdata er temperatur, nedbør eller koncentration af et bestemt stof i jorden. \\
Kategoriske værdier bruges derimod til at repræsentere diskrete eller tematiske vædier, hvor hver celle tildeles en bestemt kategori eller klasse. Dette ses fx i arealanvendelseskort, hvor hver celle angiver om området er dækket af skov eller trætyper (figur \ref{Subfig: Kategorisk raster}).
\begin{figure}[H]
    \begin{subfigure} [t]{0.5\textwidth}
        \centering
        \includegraphics[width=1\linewidth]{images/teori/raster_kontinuert.jpg}
        \caption{}
        \label{Subfig: Kontinuer raster}
    \end{subfigure}
    \begin{subfigure} [t]{0.5\textwidth}
        \centering
        \includegraphics[width=1\linewidth]{images/teori/raster_areal.png}
        \caption{}
        \label{Subfig: Kategorisk raster}
    \end{subfigure}
    \caption{Eksempel på en raster med numeriske kontinuert data fra en højdemodel \textbf{(a)} og en kategorisk raster med diskrete data over forskellige trætyper og arealanvendelser\textbf{(b)}. Kilde: \cite[s. 66]{bolstad_gis_2022} og \cite[s. 67]{longley_geographical_2008}}
    \label{Figur: Kontinuert og kategorisk raster}
\end{figure}
Det er vigtigt at bemærke at rasterceller også kan indeholde en unik værdi, der angiver 'NoData', hvis der ikke foreligger information for det pågældende område. NoData værdier gør det muligt at håndtere ufuldstændige datasæt og sikring at analyser kun udføres på relevante celler \citep{bolstad_gis_2022, longley_geographical_2008}.

Rastermodellens struktur muliggør omfattende rumlig analyse, idet der kan udføres artimetiske og logiske operationer på tværs af celler og mellem flere forskellige typer af rasterlag. Denne egenskab gør det derfor muligt at analysere og kombinere kontinuerte og diskrete rumlige informationer i et GIS-miljø  \citep{bolstad_gis_2022, longley_geographical_2008}. Det er også muligt at udføre analyse mellem raster og vektordata. Ved at udføre analyse med begge datatyper er det muligt at lave kombineret vektor- og raster modeller i et GIS-miljø der blandt andet kan bruges til at lave stormflodsmodellering.

\subsection{Inundation Modellen} \label{Afsnit: Inundation Model}

Til projektet er der blevet anvendt en GIS-baseret stormflods model kaldet \textit{"Inundation Model"} udarbejdet af \cite{balstrom_kirby_inundation} til at give et bud på hvordan en stormflod vil påvirke et område. Modellen opererer eksklusivt i et ArcGIS Pro miljø, GIS-softwaren udviklet af Esri.\\
Modellen indeholder en række værktøjer, der bruges til at analysere stormfloders påvirkning af et område og kernen i modellen er værktøjet \textit{"Create Inundation"}, en statisk numerisk rastermodel. Modellen tager tre brugerdefineret parametre der består af tre numeriske værdier i meter som input: InitialSealLevel, SeaLevelIncrement og Number of Iterations. InitialSeaLevel bruges som en startværdi for modellen til at itererer over. SeaLevelIncrement er den værdi modellen skal stige med efter hver iteration (fx 1 cm stigning). Number of Iterations er antallet af gentagelser modellen udører. \\
Modellen benytter en hydrologisk korrigeret Digital Terræn Model (DHyM) og et digitaliseret linjeobjekt som kilde for udregningen, benævt Line at Sea \citep{balstrom_kirby_inundation}. I figur \ref{Figur: Create Inundation} er modellens struktur vist inddelt i to segmenter.
\begin{figure}[H]
    \centering
    \includegraphics[width=1\linewidth]{images/teori/inundation_model_separated.jpg}
    \caption{Flowchart af Inundation Modellens \textit{"Create Inundation"} værktøj.}
    \label{Figur: Create Inundation}
\end{figure}


Det første segment er en omregning af brugerens input fra meter til centimeter for både stigningsniveauet for hver gang modellen itererer og det begyndende havniveau. Modellen kræver en afsluttende værdi for hvornår den skal stoppe med at iterere, som svarer til den vandstand der ønskes at simulere op til. Denne værdi bliver udregnet ved følgende: \\$InitialSeaLevel + ((NumberIterations - 1)\times Increment)$. På samme vis kan en omvendt udregning laves for at finde antallet af gentagelser for at opnå en bestemt vandstand. Dette gøres ved: $(EndValue - InitialSeaLevel) / Increment - 1$. Dette tillader brugeren at simulere til en ønsket vandstand.\\

Det andet segment af modellen er selve oversvømmelses beregningen gennem højdemodellen. Det starter med et for-loop der starter med den første værdi (fx 100 cm). Denne værdi omregnes tilbage til meter hvorefter modellen eksekverer et hvis-ellers tjek (figur \ref{Figur: Create Inundation} "SetNull") på cellerne i DTM. Her tjekkes alle celleværdierne i DTM for om de er større eller ligmed den pågældende værdi. Hvis dette hvis-ellers tjek er sandt når værdien i DTM er højere end tjekværdi, bliver cellerne tildelt NoData værdien, som indikerer at cellen ikke bliver oversvømmet ved dette oversvømmelsesniveau. Hvis DTM celleværdien er lavere end tjekværdien og udtrykket dermed er falsk bliver cellen angivet med et 1, som indikerer at cellen oversvømmes ved det niveau. I figur \ref{Figur: Celler Inundated} er der vist hvordan det sker for tre iterationer startende på 100 cm med en stigning på 10 cm. Her det værd at nævne at hver celle der tjekkes for ikke udelukker hinanden, så den næste værdi (110 cm) vil også tildele et 1 til de samme celler, som den foregående værdi.       
\begin{figure}[H]
    \centering
    \includegraphics[width=0.7\linewidth]{images/teori/celler_inundated.png}
    \caption{Princippet bag Inundation Modellen gennem cellerne i en højdemodel. 1 angiver at cellen oversvømmes ved det pågældende niveau og ND = NoData (celler som ikke bliver oversvømmet). Egen illustration med inspiration fra \cite{balstrom_kirby_inundation}.}
    \label{Figur: Celler Inundated}
\end{figure}
Herefter gennemføres en Distance Accumulation-analyse igennem de celler der bliver oversvømmet ved det bestemte niveau fra linjekilden Line at Sea. En Distance Accumulation er en analysemetode der beregner den samlede afstand fra en defineret kilde ud igennem et område \citep{esri_how_nodate}. I Inundation modellen bliver Distance Accumulation brugt til at efterligne vandets bevægelse igennem cellerne på samme måde som vandet ville sprede sig under en stormflod. Distance Accumulation forsætter gennem terrænet, indtil der mødes en uigennemtrængelig barriere. Til at sprede sig igennem terrænet bruges der en "eight-side" \hspace{0.2cm}regel, der er med til at bestemme hvilke celler der kan spredes til. Eight-side reglen fortæller Distance Accumulation værktøjet at der skal tjekkes for en passerbar celle i otte retninger fra en celle.\\

I Inundation Modellen starter Distance Accumulation fra linjen \textit{"Line at Sea"} og bevæger sig igennem alle cellerne i terrænet hvor cellen = 1. Hvis cellen er NoData så kan vandet ikke bevæge sig igennem \citep{balstrom_kirby_inundation}. Resultatet af Distance Accumulationen bliver derefter koblet med det oversvømmelsesniveau der bliver itereret over for at give resultatet af modellen og processen starter derefter forfra indtil slutværdien gennemføres.  


\subsection{Stormfloder} \label{Afsnit: Stormfloder}

En stormflod er betegnelsen for usædvaneligt højvande i relation til kraftig vind \citep{shoreline_management_guidelines}.
Styrken og påvirkningen af en stormflod på en lokalitet er afhængig af en række faktorer herunder orienteringen af kysten i forhold til stormen, hvor kraftig stormen er og de lokale havbunds- og dybdeforhold \citep{noaa_storm, shoreline_management_guidelines}.\\

I Danmark er det Vadehavskysten der ofte er det mest udsatte område, da vinden i Danmark primært kommer fra vest \citep{cappelen_dmi_2020}. Over længere perioder med vesten vind, vil vandet i Nordsøen blive presset ind Kattegat og længere ned i de indre danske farvande. Hvis vinden er langvarig, vil vandet over tid blive presset igennem bælterne ved Lillebælt, Storebælt og Øresund og videre ind i Østersøen og nord imod den Botniske Bugt. \citep{kystdirektoratet_stormfloder}. Dette fænomen kaldes for \textit{"preconditioning"} og beskriver hvordan vandstanden stiger i Østersøen inden begyndelsen af en storm \citep{kiesel_brief_2024, weisse_sea_2021}. \\   

Når vinden derefter aftager eller ændrer retning vil alt vandet der er blevet presset ind i Østersøen, skvulpe tilbage mod bælterne i Danmark. Dette kaldes for \textit{"badekarseffekten"} og er illustreret i figur \ref{Figur: Bathtub effect} \citep{kystdirektoratet_stormfloder, egusphere_baltic}. Bælterne i de indre danske farvande vil herefter fungere som flaskehalse for vandmasserne der skvulper tilbage fra Østersøen og resultere i oversvømmelse i de indre danske farvande \citep{egusphere_baltic}.
\begin{figure}[H]
    \centering
    \includegraphics[width=0.8\linewidth]{images/teori/bathtub effect graphics.jpg}
    \caption{Illustration af "badekarseffekten". De sorte pile indikerer vindretning og de blå og røde pile indikerer bevægelsen af vandmassen. Kilde: Egen illustration, baggrundskort fra Esri}
    \label{Figur: Bathtub effect}
\end{figure}
Nævneværdige stormfloder som stormfloden den 1.-2. november 2006, den 20.-21. oktober 2023 og den 12.-14. november 1872 blev forårsaget af kraftig østenvind og \textit{"badekarseffekten"} \hspace{1.5cm} \citep{kystdirektoratet_stormfloder}.

\subsection{Stormfloden den 20 oktober 2023} \label{Afsnit: Stormfloden den 20 oktober 2023}
I første halvdel af oktober 2023 blev der observeret moderate vindforhold med gennemsnitlig middelvind på 5,5 m/s og en gennemsnitlig maksimal 10.min middelvind på 18,3 m/s fra vest, hvilket medførte en betydelig vandtransport ind gennem Kattegat og videre i Østersøen \citep{dmi_vejrarkiv}. \\
Den 18. oktober skiftede vindretningen til øst (figur \ref{Figur: Vinddata Danmark}) grundet trykforskelle mellem et højtryksystem over Skandinavien og et lavtrykssystem over Storbritannien \citep{kiesel_brief_2024}, og middelvindhastigheden steg i hele landet til 12,2 m/s og maksimale 10.min middelvind til 28,3 m/s om aftenen den 20. oktober 2023. 
\begin{figure} [H]
    \centering
    \includegraphics[width=0.8\linewidth]{images/vinddata_grafer/Danmark_vinddata.pdf}
    \caption{Vinddata for Danmark i oktober 2023 der viser middelvind, højeste 10.min middelvind og højeste vindhastighed i m/s. De sorte pile i bunden indikerer vindretningen. Kilde: Data fra \cite{dmi_vejrarkiv} }
    \label{Figur: Vinddata Danmark}
\end{figure}

Kombinationen af kraftig vind fra øst og stor vandtransport ind i Østersøen over en periode på 15 dage resulterede ifølge \cite{kystdirektoratet_stormflod2023} i markante vandstandsstigninger i de indre farvande af Syddanmark den 20. oktober. Den maksimale vandstand målt under stormfloden blev målt i Aabenraa Havn på 2,16 m over dagligt vande \citep{damberg_vaerste_2023}. Disse forhold forårsagede omfattende skader af sommerhusområder, kystnære områder og byer \citep{kystdirektoratet_stormflod2023, naturskaderadet_anmeldelser_2023}








\section{Studieområder}

I dette projekt er der blevet undersøgt hvordan stormfloden påvirkede fire studieområder i Danmark. De fire områder er Aabenraa, Gedser Havn, Hesnæs og Præstø.\\

Til udvælgningen af studieområderne blev i alt otte kommuner kontaktet. Dette inkluderer Aabenraa, Faaborg-Midtfyn, Guldborgsund, Haderslev, Stevns, Sønderborg og Vordingborg. Udpegningen af potentielle studieområder skete på baggrund af en tabel fra \cite{damberg_vaerste_2023} over højeste målte vandstande i Danmark under 2023-stormfloden. 
Ud af de otte kommuner blev tre kommuner (Aabenraa, Guldborgsund og Vordingborg) udvalgt til videre samarbejde på baggrund af tilgængeligt data. Guldborgsund kommune stillede data fra fire lokaliteter i kommunen til rådighed og der blev udvalgt de to kystbyer, Gedser Havn og Hesnæs, på baggrund af højere målt vandstand under stormfloden.\\

De fire udvalgte studieområder er lokaliseret langs sydkysten af Danmark. På figur \ref{Figur: Oversigtskort} er områdernes placering i Danmark visualiseret.
\begin{figure}[H]
    \centering
    \includegraphics[width=1\linewidth]{images/studieområder/oversigtskort.jpg}
    \caption{Oversigtskort over studieområderne: Aabenraa, Gedser Havn, Hesnæs og Præstø. Baggrundskort fra ESRI og Klimadatastyrelsen.}
    \label{Figur: Oversigtskort}
\end{figure}

{\large Aabenraa}\\
Aabenraa er en by i det sydøstlige Sønderjylland. Byen er beliggende i Aabenraa kommune for enden af den ca. 10 km lange Aabenraa fjord. Byen har en befolkning på 16500 og er den niende største by i Region Syddanmark \citep{danmarks_statistisk_mobile_nodate}. Byen består af to industri havne, Aabenraa og Ensted, og begge er vigtige for skibstrafik med 380 anløbte skibe i 2022 \citep{aabenraa_havn_aabenraa-havn-talogfakta2022_2022}.\\
Aabenraa rapporterede den højeste officielle måling af vandstand under stormfloden på 2,16 meter over DVR90 (figur \ref{Subfig: Aabenraa vandstand}). \\

{\large Gedser Havn}\\
Gedser Havn er en by på sydspidsen af Falster i Guldborgsund kommune. Gedser Havn fungerer som en aktiv fiskerihavn og som en færgehavn til den tyske havn Warnemünde ved Rostock. Byen har et indbyggertal på 670 i 2024 \citep{danmarks_statistisk_mobile_nodate} og er Danmarks sydligste by. \\
Under stormfloden blev der observeret en vandstandsstigning på 1,89 meter over DVR90 i Gedser Havn (figur \ref{Subfig: Gedser vandstand}), den højeste vandstand målt siden 1892, og indstillede færgetrafikken til Tyskland frem til formiddagen den 21. oktober \citep{tiirikainen_sadan_2023}.\\
\begin{figure}[H]
    \begin{subfigure}[b]{0.5\textwidth}
        \centering
        \includegraphics[width=1\textwidth]{images/studieområder/vandstands_grafer/vandstand_aabenraa_vandstandsplot.jpg}
        \caption{}
        \label{Subfig: Aabenraa vandstand}
    \end{subfigure}
    \hspace{0.2cm}
    \begin{subfigure}[b]{0.5\textwidth}
        \centering
        \includegraphics[width=1\textwidth]{images/studieområder/vandstands_grafer/vandstand_gedser_vandstandsplot.jpg}
        \caption{}
        \label{Subfig: Gedser vandstand}
    \end{subfigure}
    \vspace{0.2cm}
    \begin{subfigure}[b]{0.5\textwidth}
        \centering
        \includegraphics[width=1\textwidth]{images/studieområder/vandstands_grafer/vandstand_hesnaes_vandstandsplot.jpg}
        \caption{}
        \label{Subfig: Hesnæs vandstand}
    \end{subfigure}
    \hspace{0.2cm}
    \begin{subfigure}[b]{0.5\textwidth}
        \centering
        \includegraphics[width=1\textwidth]{images/studieområder/vandstands_grafer/vandstand_praestoe_roedvig_vandstandsplot.jpg}
        \caption{}
        \label{Subfig: Rødvig vandstand}
    \end{subfigure}
    \caption{Officiel målt vandstandsniveau i Aabenraa, Gedser Havn, Hesnæs og Rødvig fra den 15. til 23. oktober. \textbf{(a)} Vandstanden i Aabenraa fra 15. til 23. oktober 2023. \textbf{(b)} Vandstanden i Gedser Havn fra 15. til 23. oktober 2023. \textbf{(c)} Vandstanden i Hesnæs fra 15. til 21. oktober 2023. \textbf{(d)} Vandstanden i Rødvig fra 15. til 23. oktober 2023. Kilde: Data stammer fra DMI og DMIs Frie Data API-ressource.}
    \label{Figur: Vandstandsdata}
\end{figure}
{\large Hesnæs}\\
Hesnæs er et lille fiskerleje og landsby på det østlige Falster i Guldborgsund kommune. Hesnæs fungerer dagligt som en fiskeri- og lystbådehavn og er den eneste havn på Falsters østkyst. Hesnæs er beliggende inde mellem to skove: Corselitze skov mod nord og Bønnet skov mod syd, som begge går ud til skrånede klinter ud til Østersøen og danner et naturligt lavtliggende område ved landsbyen. \\
Hesnæs blev særdeles hårdt ramt af stormfloden, hvor der blev officielt målt 2,10 m vandstand over DVR90. Uofficielle målinger nåede op på 2,39 meter over DVR90, men vandstandsmåleren gik i stykker kort tid efter kl. 20:30 den 20. oktober (figur \ref{Subfig: Hesnæs vandstand}). Meget af Hesnæs havn blev svært beskadiget af stormfloden på grund af Hesnæs eksponering og større fristræk mod Østersøen. Især den ydre mole af havnen blev ødelagt af høje bølger og broerne i havnen blev skyllet på land. Lystbådehavnen er på nuværende tidspunkt stadig ude af drift og skaderne på havnens mole er stadig synlige.\\

{\large Præstø}\\
Præstø er en havneby og tidligere købstad i Vordingborg Kommune på Sjælland. Byen er beliggende i den sydlige del af Præstø Fjord, bag halvøen Feddet i bunden af Faxe Bugt. Tværs igennem byen løber Tubæk Å, et vandløb der løber fra tunneldalen Tubæk lidt syd for byen. Tubæk Å deler Præstø mellem et nordligt handelscentrum og et sydligt boligkvarter. Byen har en befolkningstal på ca. 4000 i 2025 \citep{danmarks_statistisk_mobile_nodate} og er kommunens andenstørste by.\\
Under stormfloden blev store dele af Præstøs nordlige handelscentrum oversvømmet, især efter sluseporten der bruges til at regulere vandniveauet i Tubæk Å, brød sammen \citep{uldall_sluseport_2023}. Der er ingen officielle målinger om vandstandshøjden i Præstø under stormfloden, da den nærmeste målestation er placeret længere opstrøms i Tubæk Å. \cite{cowi_praesto_2025} giver et skøn på at vandstanden var ca. 40 cm højere end i Rødvig Havn i Stevns Kommune ca. 25 km nordøst fra Præstø. Dette giver en vandstand på ca. 2 meter over DVR90 under stormfloden i Præstø (figur \ref{Subfig: Rødvig vandstand}).

\section{Metoder}


Der vil i dette afsnit blive beskrevet den metodiske tilgang til projektets analyse og resultater. Der vil først blive beskrevet de dataprodukter der er anvendt og det databehandlinger der er blevet udført. Derefter beskrives arbejdsgangen bag brugen af Inundation Modellen og hvordan den fremskrevne stormflod er beregnet. Metodiske fravalg og fejlkilder vil blive diskuteret i sektion \ref{Resultat Diskussion}. 

\subsection{Databeskrivelse og databehandling} \label{Sektion: Databeskrivelse}
% god teknik er at lave en label til sektionen, så den er nem at referere til (just in case).


I den følgende sektion vil der blive beskrevet det data der er blevet anvendt i projektet. Herudover vil der også blive beskrevet hvordan data eventuelt er blevet ændret i forhold til det originale data. 

\subsubsection{Digital Terrænmodel} \label{Afsnit: Digital Terræn Model}
Modellering af oversvømmelser fra havet kræver en digital terrænmodel (DTM). En DTM er en digital repræsentation af højderne i landskabet i forhold til en reference. En DTM adskiller sig fra en digital overflademodel (DSM) ved ikke at inkludere bygninger og træer \citep{sdfe_dhm_2020}. En DTM bliver lavet med LiDAR flyscanning og optages i Danmark over en 5-årig periode i forskellige sektioner. 
I Danmark er det Klimadatastyrelsen (\textit{tidl.} Styrelsen for Digitalisering og Infrastruktur), som står for etableringen af DTM \citep{sdfe_dhm_2020}. \\

I dette projekt er der anvendt den seneste opdateret DTM fra 2023. For Aabenraa, Gedser og Hesnæs er DTM senest optaget i 2023, mens DTM for Præstø er senest optaget i 2019. DTM for Danmark lagres i et GeoTIFF rasterformat med en cellestørrelse på 0,4$\times$0,4m (0,16 m\textsuperscript{2}). DTM er i en UTM zone 32 nord projektion og i ETRS89 koordinatsystemet med en horisontal nøjagtighed på \pm 15 cm. Den vertikale reference er i DVR90 med en vertikal nøjagtighed på \pm 5 cm \citep{sdfe_dhm_2020}. \\
For at minimere processeringstid, blev DTM skåret ned til en afgrænsning af hvert studieområde. Afgræsningen skete på baggrund af studieområdets omkringliggende topografi og udstrækket af eventuel byzone. Derudover er DTM blevet konverteret til en digital hydrologisk terrænmodel (DHyM) og bygningspolygoner er blevet brændt ned som ikke passerbare enheder i terrænet. Fremgangsmåden for dette er beskrevet i henholdsvis sektion \ref{Sektion: Konvertering af DTM til DHyM} og \ref{Afsnit: Inklusion af bygninger i DHyM}.


\subsubsection{BaseMap arealanvendelsesdata} \label{Afsnit: Arealanvendelses data}
For at undersøge 2023-stormflodens påvirkning af de udvalgte områder, anvendes der et arealanvendelses datasæt til at kvantificere de påvirkede arealanvendelser under stormfloden. \\
Dette blev gjort ved at bruge det danske arealanvendelses datasæt BaseMap produceret af \cite{Jepsen_levin_2013}. BaseMap dækker omtrent 98\% af Danmarks areal med 35 forskellige arealanvendelsesklasser og er leveret i et rasterformat med en cellestørrelse på 10\times10 m. Datasættet er senest opdateret i 2022 med version fire og det er denne version der er anvendt \citep{levin_basemap04_2022}.\\

For at forsimple visualiseringen af påvirkede arealanvendelser er der blevet udført en reklassifikation af arealklasserne på samme måde som \cite{balstrom_kirby_inundation}. Reklassfikationen trimmede datasættet fra 35 oprindelige arealklasser til 13 overordnede klasser. Under reklassificeringen er en række af klasserne herunder lufthavn, råstofudvinding og Tyskland, som ikke er tilstede i studieområderne, blevet ekskluderet. Klasserne hav, vandløb og søer er også blevet fjernet, da de ikke har en relevans for undersøgelsen.\\ 
Dette resulterer i 8 overordnede klasser der er vist i tabel \ref{Tabel: arealanvendelses klasser} samt hvilke klasser fra BaseMap der indgår i de nye aggregerede arealklasser.
\begin{table}[H]
\centering
\renewcommand{\arraystretch}{1.5}
\begin{threeparttable}
\caption{Reklassificerede arealklasser baseret på \cite{balstrom_kirby_inundation} og hvilke arealklasser fra Basemap \citep{Jepsen_levin_2013} der indgår i hver klasse.}
\label{Tabel: arealanvendelses klasser}
\begin{tabular}{@{} l l l @{}} 
\toprule
\textbf{ID} & \textbf{Aggregerede klasser} & \textbf{BaseMap04 klasser} \\
\midrule
1 & Bebyggede områder &
  \makecell[l]{Bygning, Lav bebyggelse, Lav bebyggelse; Bygning,\\
  Høj bebyggelse, Høj bebyggelse; Bygning,\\
  Bykerne, Bykerne; bygning, Andet bebyggelse,\\
  Andet bebyggelse; Bygning} \\ 
  \addlinespace
2 & Erhverv &
  \makecell[l]{Erhverv, Erhverv; Bygning} \\
  \addlinespace
3 & Rekreativt &
  \makecell[l]{Rekreativt område / sportsanlæg,\\
  Rekreativt område / sportsanlæg; Bygning} \\
  \addlinespace
4 & Infrastruktur &
  \makecell[l]{Vej; befæstet, Vej; ikke befæstet,\\
  Jernbane, Jernbane; Bygning} \\
  \addlinespace
5 & Landbrug &
  \makecell[l]{Landbrug intensivt; midlertidige afgrøder,\\
  Landbrug intensivt; permanente afgrøder,\\
  Landbrug ekstensivt, Landbrug; ikke klassificeret} \\
  \addlinespace
6 & Skov &
  \makecell[l]{Skov, Skov; Våd} \\
  \addlinespace
7 & Naturområde &
  \makecell[l]{Natur; tør, Natur tør; Landbrug ekstensivt,\\
  Natur; våd, Natur våd; Landbrug ekstensivt} \\
  \addlinespace
8 & Uklassificeret &
  \makecell[l]{Ikke kortlagt} \\
\bottomrule
\end{tabular}
\end{threeparttable}
\end{table}


\subsubsection{Hydrologiske tilpasninger} \label{Afsnit: Hydrologiske tilpasninger}
I processen for at konvertere en DTM til DHyM er der blevet anvendt to hydrologiske tilpasningslag: Linje- og hesteskotilpasninger. Tilpasningerne er begge et-dimensionelle geometriske dataobjekter defineret af GeoDanmark som linjer \citep{GeoDanmark_HydroLag}. \\
Linjetilpasningerne er en defineret som et enkelt linjeobjekt, der beskriver en åbning for overfladevands forløb gennem en hindring eller en hindring af overfladevands forløb gennem et terræn \citep{DHMLinje}. 
Linjetilpasningerne er den simpleste hydrologiske tilpasning og forekommer ved bl.a. rør og små vandløb der passerer under veje eller mellem marker. I figur \ref{Subfig: Linjetilpasning} er det et eksempel på en linjetilpasning gennem et rør fra en mark ud til havet.\\

Hesteskotilpasningerne er defineret af \cite{DHM_Hestesko}, som et hestesko-formet geometrisk objekt der tillader eller begrænser overfladevandets forløb gennem en hindring eller gennem terrænet. Bredden af hesteskoen definerer hindringens størrelse. Hesteskotilpasningerne anvendes ved hindringer i landskabet der er større end et mindre vandløb eller rør og dermed kræver en mere korrekt repræsentation i landskabet. Hesteskotilpasningerne findes under større broer og tunneller under veje og i figur \ref{Subfig: Hesteskotilpasning} er der vist et eksempel på en hesteskotilpasning fra en cykeltunnel under en vej.
\begin{figure}[H]
    \begin{subfigure}[b]{0.5\textwidth}
        \centering
        \includegraphics[width=1\linewidth]{images/databeskrivelse/linje.jpg}
        \caption{}
        \label{Subfig: Linjetilpasning}
    \end{subfigure}
    \hspace{0.2cm}
    \begin{subfigure}[b]{0.5\textwidth}
        \centering
        \includegraphics[width=1\linewidth]{images/databeskrivelse/hestesko.jpg}
        \caption{}
        \label{Subfig: Hesteskotilpasning}
    \end{subfigure}
    \caption{Eksempler på en \textbf{(a)} Linjetilpasning fra et mindre vandløb under en græsplæne. \textbf{(b)} Hesteskotilpasning fra en cykeltunnel under en vej i Aabenraa.}
    \label{Figur: Linje- og hesteskotilpasninger}
\end{figure}


\subsubsection{Vind- og vandstandsdata} \label{Vind- og vandstandsdata}
Til at forstå konteksten bag stormflods hændelsen den 20-21. oktober 2023 er der blevet anvendt vind-og vandstandsdata for måneden oktober 2023. \\
Vinddata er indsamlet fra Danmarks Meteorologiske Instituts (DMI) vejrarkiv \citep{dmi_vejrarkiv} for hele landet og er der beregnet et gennemsnit for middelvind, højeste 10-min middelvind og den maksimale vindhastighed i m/s for hver dag i måneden. \\

Vandstandsdata er indsamlet på to måder. Data fra Gedser, Hesnæs og Præstø er indsamlet fra DMI's Open Data API-Oceanographic Observation Data ressource \citep{dmi_open_data}. Vandstandsdata er optaget hvert 10. minut fra den 1. oktober til den 31. oktober 2023 og er herefter filtreret til kun at vise data fra den 15. til den 23. oktober. Data for Præstø er hentet fra Rødvig Havn, ca. 25 km nordøst, da Præstø ikke har en officiel DMI vandstandsmåler. Dette er den samme station, som \cite{cowi_praesto_2025} har anvendt til at give et skøn på vandstanden målt i Præstø. \\
Vandstandsdata fra Aabenraa Havn er indsamlet fra en samtale med DMI med tilladelse fra virksomheden Aabenraa Havn, da DMI ikke stiller havnens rå vandstandsmålinger til rådighed for offentligheden.\\

Vandstandsdata fra Gedser Havn havde en del afvigelser, hvor vandstanden svingede fra +200 cm til -340 cm på 10 minutter. De datapunkter er fjernet efter et kriterie om at vandstanden ikke realistisk ville kunne ændres med mere end \pm 25\% på 10 minutter. Hvis vandstanden ikke er indenfor grænsekriterieret, så fjernes datapunktet og det efterfølgende datapunkt tages i betragtning. Hvis det næste datapunkt er indenfor grænsen, så bliver det datapunkt den nye værdi der tjekkes. Logikken bag filtreringen er vist i ligning \ref{Eq: Outlier vandstand}. Hvor $x$ er et datapunkt i serien og $x_i$ er det næste datapunkt i serien.
\begin{align} \label{Eq: Outlier vandstand}
    \text{Hvis } x_i \in [0.75\times x, 1.25\times x], \text{ så } x \leftarrow x_i \nonumber \\
    \text{Hvis } x_i \notin [0.75\times x, 1.25\times x], \text{ så fjernes } x_i
\end{align}

\subsubsection{Data fra studieområderne og andet anvendt data} \label{Afsnit: Data fra studieområderne og andet anvendt data}
Projektet benytter sig af data modtaget fra de tre kommunerne hvor undersøgelsen finder sted. Den primære dataform er et raster kort over vandets udbredelse fra stormflodshændelsen i 2023. Alle kortene blev resamplet til en cellestørrelse på 0,4\times0,4 m, for at sikre samme cellestørrelse som DTM, da alle kortene havde forskellige cellestørrelser der variede mellem 0,45 og 0,5 m.\\
Aabenraa kommune har derudover givet information, linje- og punktdata omkring beredskabsindsatser, herunder placeringerne af watertubes og dronebilleder over byen.\\ 

Udover det ovennævnte data er der blevet anvendt en række mindre dataprodukter. 
Dette inkluderer et datasæt med bygningspolygoner fra GeoDanmark, der er anvendt til at brænde bygninger ned i DHyM. I sektion \ref{Afsnit: Inklusion af bygninger i DHyM} er der beskrevet hvordan bygningspolygon datasættet er blevet ændret. En topografisk landpolygon over Danmark fra Klimadatastyrelsen, er anvendt som en afgrænsningsmaske til Inundation Modellens resultat. Et datasæt over vandstandsstignings projektioner for 2030 fra den sjette \cite{ipcc_report_AR6, garner_ipcc_2021} klimarapport samt et værktøj med stationsnumre for projektionerne fra \cite{NASA_tool}. Dette datasæt anvendes til at beregne en fremskrevet stormflod i 2100.



\subsection{Konvertering af DTM til DHyM} \label{Sektion: Konvertering af DTM til DHyM}
Et vigtigt element af en realistisk oversvømmelsesmodellering er at kunne simulere korrekt hydrologisk adfærd gennem terrænet. Det er derfor essentielt for at kunne bruge modellens resultater tillidsfuldt at den digitale terrænmodel (DTM) bliver korrigeret og konverteret til en digital hydrologisk model (DHyM).\\ 

For at konvertere en DTM til DHyM bliver der anvendt to geometrisk dataobjekter: Linje- og hesteskotilpasninger. Begge objekter er linjeobjekter og bliver brændt ned i DTM for at simulere korrekt geografiske karakteristika i landskabet, som ikke er blevet fanget af højdemodellen. \\
Da der findes tilpasninger for både simuleringer af skybrud og stormfloder, er det vigtigt at der kun bliver anvendt de tilpasninger der har indflydelse på havstigningsmodellering. Det vil derfor betyde at der ikke skal inkluderes tilpasninger, som er designet til at føre regnvand fra skybrud ud i havet. Tilpasninger såsom skybrudskontraklappe er derfor blevet fravalgt. Der er derfor lavet en selektering af de korrekte tilpasninger i databasen for hvert område.\\

Dette er gjort for både linje- og hesteskotilpasningerne ved brug af værktøjet \textit{"Extract Hydroconditoning Inundation"}, som laver en søgning i databaserne for tilpasningslagene. Først søger værktøjet efter alle tilpasninger indenfor studieområdet defineret ud fra en maske. Herefter laves der en søgning i attributtabellen for en anvendelses beskrivelse. Til stormflodsmodellering skal der anvendes tre attributter og deres funktionalitet er beskrevet i tabel \ref{Tabel: Relevante hydrologiske tilpasninger}. De tilpasninger som opfylder én af kriterierne bliver overført til et nyt lag, en for linjetilpasninger og en for hesteskotilpasninger.  

\begin{table}[H]
\centering
\renewcommand{\arraystretch}{1.5}
\begin{threeparttable}
\caption{Relevante hydrologiske tilpasninger. Kilde: \cite{GeoDanmark_HydroLag}}
\label{Tabel: Relevante hydrologiske tilpasninger}
\begin{tabular}{@{} l l @{}} 
\toprule
\textbf{Navn} & \textbf{Beskrivelse} \\
\midrule
\textit{Generel} &
  \makecell[l]{Den normale tilpasning af hydrologiske forhold\\
  (fx skabelse af et frit forløb under en bro)} \\
\addlinespace
\textit{Havstigning} &
  \makecell[l]{Tilpasninger der skal forhindre, at vand løber ind\\
  over det bagvedliggende land\\
  (fx lukning af en højvandssluse)} \\
\addlinespace
\textit{DHMFix} &
  \makecell[l]{Bruges ved ændringer, der har hydrologisk effekt\\
  på vandets frie forløb på terrænoverfladen\\
  (fx reparation af fejl i den specifikke DHM)} \\
\bottomrule
\end{tabular}
\end{threeparttable}
\end{table}

Inden tilpasningerne bliver brændt ned i DTM, skal hesteskotilpasningerne ændres fra deres hesteskoform til at have en række linje mellem ydrekanterne af hesteskoen. Dette gøres for at få den præcise størrelse og udbredelse af tilpasningen brændt ned i DTM fremfor for omridset af tilpasningen. Det er blevet gjort ved at bruge et Python-script \textit{"Convert Horsehoes to Lines"} hvor både linje- og hesteskotilpasningerne fundet i \textit{"Extract Hydroconditioning Inundation"} bliver omdannet til separate linjer. Ved at konvertere hesteskotilpasningerne til separate linjer er det muligt at brænde tilpasningens korrekt form ned i DTM, da hesteskotilpasningernes originale form ikke er egnet til at blive brændt ned i DTM. \\
Linjerne til hesteskotilpasningerne bliver defineret ud fra de 4 hjørnepunkter af hesteskoen og cellestørrelsen af DTM. Ændringen af hesteskotilpasningen til linjer er vist i \ref{Figur: Ændringen af hesteskotilpasningerne}. Hvor figur \ref{Subfig: Hesteskotilpasninger før ændring} er tilpasningen før ændring og figur \ref{Subfig: Hesteskotilpasning efter tilpasningen er konverteret til linjer} er tilpasningen efter den er blevet konverteret til separate linjer. Hver linje er præcis en cellestørrelse bred, da det ikke er nødvendigt at have en større linjebredde for at simulere korrekt hydrologisk bevægelse igennem terrænet. 

\begin{figure}[H]
    \begin{subfigure}[t]{0.5\textwidth}
        \centering
        \includegraphics[width=1\linewidth]{images/databeskrivelse/hestesko.jpg}
        \caption{Hesteskotilpasninger før ændring}
        \label{Subfig: Hesteskotilpasninger før ændring}
    \end{subfigure}
    \hspace{0.2cm}
    \begin{subfigure}[t]{0.5\textwidth}
        \centering
        \includegraphics[width=1\linewidth]{images/metode/hestesko_linjer.jpg}
        \caption{Hesteskotilpasning efter tilpasningen er konverteret til linjer}
        \label{Subfig: Hesteskotilpasning efter tilpasningen er konverteret til linjer}
    \end{subfigure}
    \caption{Ændringen af hesteskotilpasningerne fra en hesteskoform til separate linjer}
    \label{Figur: Ændringen af hesteskotilpasningerne}
\end{figure}

Dette skaber et nyt sammenlagt lag med alle de relevante hydrologiske tilpasninger som linjer der blev brændt ned i DTM. For at tilpasningen bliver korrekt repræsenteret i DHyM skal højdeværdier under terræn hindringen først interpoleres.\\
Dette blev gjort ved at bruge et Python-script \textit{"Hydrologic Conditioning Multiple"}, som tildeler en z-værdi fra DTM til linjernes endepunkter: $Z_0$ og $Z_1$. Ud fra linjernes endepunkter bliver linjens hældning, $\Delta{Z}$ udregnet ved at bruge linjens længde. Linjen konverteres til et rasterformat og derefter til punkter for hver celle i DTM, hvor der laves en lineær interpolation af punkterne på linjen baseret på startpunktet $Z_0$, afstanden fra $Z_0$ til punktet og linjetilpasningens hældning, $\Delta{Z}$. Dette gøres for alle punkter langs linjen (figur \ref{Figur: Interpolation af Z-værdier}). \\
Efter punkterne er blevet interpoleret, konverteres der tilbage til et rasterformat, hvorefter de hydrologiske tilpasninger brændes ned i DTM ved at kombinere rasteren med de interpoleret punkter og DTM. Dette skaber den korrigeret Digitale Hydrologiske Model (DHyM).

\begin{figure}[H]
    \centering
    \includegraphics[width=0.5\linewidth]{images/metode/dtm_hydro_z.jpg}
    \caption{Interpolationen af z-værdier for en linjetilpasning under en hindring. $Z_1$ og $Z_0$ er z-værdierne fra DTM ved linjetilpasningens endepunkter og $\Delta{Z}$ er hældningen af linjetilpasningen. Egen illustration med inspiration fra \cite{balstrom_identification_2024}}
    \label{Figur: Interpolation af Z-værdier}
\end{figure}

For studieområderne Aabenraa, Gedser Havn, Hesnæs og Præstø blev der identificeret henholdsvis 250, 13, 34 og 19 hydrologiske tilpasninger, som blev brændt ned i deres tilhørende DTM.


\subsubsection{Inklusion af bygninger i DHyM} \label{Afsnit: Inklusion af bygninger i DHyM}

Efter de hydrologiske tilpasninger blev brændt ned i DTM, blev der besluttet at inkludere bygninger i DHyM, som en ikke-gennemtrængelige barrierer i terrænet. Denne beslutning blev truffet med udgangspunkt i, at de oversvømmelseskort studiekommunerne havde leveret, behandlede bygninger som barrierer. Desuden blev det vurderet at en realistisk simulering af hydrologisk adfærd forudsætter, at vand ikke kan strømme igennem bygninger, men ledes udenom. \\

Implementeringen af bygninger i DHyM blev gennemført ved at tildele alle bygningspolygoner en arbitrær højdeværdi på 20 meter. Herefter blev bygningspolygonerne konverteret til et rasterformat med den samme cellestørrelse på 40\times40 cm som DHyM. \\
Derefter udføres der et tjek på den nye raster med bygningerne, hvor celler med værdien 20 forblev 20 og celler med NoData blev tildelt værdien 0. Dette sikrer at bygningsrasteren kan kombineres med DHyM, da beregninger på NoData-værdier ikke er mulige. Til sidst blev bygningsrasteren og DHyM lagt sammen, hvilket skabte den endelige DHyM, som anvendes i Inundation Modellen.


\subsection{Simulering af oktober 2023 stormfloden}\label{Afsnit: Simulering af stormflod 2023}

Efter DTM er blevet konverteret til en DHyM er det muligt at udføre stormflodsmodellering via Inundation Modellen. I modellen inputtes der for hvert studieområde den tilhørende DHyM og kildelinjen \textit{"Line at Sea"}. Line At Sea linjen er manuelt digitaliseret i hvert studieområde ude i havet.\\ Hvert studieområde simuleres op til det højeste officielle vandstandsniveau målt ved oktober 2023 stormfloden og antallet af iterationer for at opnå dette niveau er vist i tabel \ref{Tabel: Antal iterationer og slutværdier for Inundation Model}. Alle studieområder starter oversvømmelsessimuleringen med værktøjet \textit{"Create Inundation"} fra  100 cm og en vandstandsstigningsværdi på 1 cm. En startværdi på 100 cm blev valgt på baggrund af processeringstid og vandstandsstigningsværdien på 1 cm blev valgt for at opnå størst fleksibilitet og for at få nøjagtig samme vandstandsniveau som målt under stormfloden den 20. oktober 2023. Simuleringstiden for hver af de fire studieområder er vist i tabel \ref{Tabel: Antal iterationer og slutværdier for Inundation Model} og sammenlagt tog det 12 timer og 1 minut at simulere oktober 2023 stormfloden ved brug af Inundation Modellen på de fire studieområder.  \\
\begin{table}[H]
\centering
\renewcommand{\arraystretch}{1}
\begin{threeparttable}
\caption{Antal iterationer, slutværdien og processeringstiden for oversvømmelsessimuleringer i Inundation Modellen for simulere oktober 2023 stormfloden}
\begin{tabular}{@{} l 
                S[table-format=7.2, output-decimal-marker={,}] 
                S[table-format=7.2, output-decimal-marker={,}]
                l @{}} 
\toprule
\textbf{Lokalitet} & \textbf{Antal iterationer} & \textbf{Slutværdi (cm)}  & \textbf{Processeringstid}\\
\midrule
Aabenraa & 117 & 216 & 5 timer 27 minutter \\
Gedser & 90 & 189 & 3 timer 25 minutter\\ 
Hesnæs & 111 & 210 & 1 time 23 minutter \\
Præstø & 101 & 200 & 1 time 44 minutter \\
\bottomrule
\end{tabular}
\label{Tabel: Antal iterationer og slutværdier for Inundation Model}
\end{threeparttable}
\end{table}

Efter simuleringen er gennemført er der produceret tre rasterlag, et "Inundation", "Backdirection", og "Distance Accumulation" -lag, for hver cm op til slutværdien for hvert område. Ved brug af det interne ArcGIS Pro værktøj \textit{"Cell Statistics"} blev "Inundation" \hspace{0.2cm} lagene fra 100 cm til slutværdien for det givne studieområde kombineret ved at finde minimum oversvømmelseværdi af hver celle. Her er der blevet brugt en landpolygon fra \cite{klimadatastyrelsen_landpolygon} til at filtrere resultatet, så det kun er oversvømmelsen på land der er visualiseret. Dette producerer et samlet kort over oversvømmelsen simuleret af Inundation Modellen der kan sammenlignes med de modtaget kort over vandets udbredelse fra studiekommunerne. 

\subsection{Kvantificering af påvirkede areal anvendelser} \label{Afsnit: Udregning af påvirkede areal anvendelser}

Til at kvantificere påvirkningen 2023 stormfloden havde på studieområderne blev der gennemført en krydstabulering mellem arealanvendelsesklasserne fra BaseMap04 og både den målte og simuleret oversvømmelse af stormfloden i oktober 2023. \\

Først blev BaseMap04-arealanvendelsesrasteren, der oprindeligt har en cellestørrelse på 10\times10 m, resamplet til en cellestørrelse svarende til outputtet fra Inundation Modellen (40\times40 cm). Derefter blev værktøjet \textit{"Tabulate Area"} anvendt til at beregne det oversvømmede areal for hver arealanvendelsesklasse, både for den målte 2023-hændelse og for resultaterne fra Inundation Modellen. 
\begin{figure}[H]
    \centering
    \includegraphics[width=1\linewidth]{images/metode/tabulate.jpg}
    \caption{Krydstabulerings processen bag Tabulate Area værktøjet i ArcGIS Pro. Hver celle i oversvømmelsesrasteren kobles til en celle i arealanvendelsesrasteren og tabuleres. Egen illustration med inspiration fra \cite{esri_tabulate_nodate}}
    \label{Figur: Tabulate}
\end{figure}
Værktøjet Tabulate Area krydsreferer hver celle i BaseMap-arealanvendelsesrasteren med den tilsvarende celle i oversvømmelsesrasteren så der bliver knyttet en arealanvendelse til et oversvømmelsesniveau, der muliggør en kvantificering af stormflodens påvirkning af studieområderne (figur \ref{Figur: Tabulate}). \\

Resultatet af krydstabuleringen præsenteres som en kontingenstabel med arealklasserne som rækker og hvert oversvømmelsesniveau i centimeter som kolonner. På baggrund af denne tabel summeres antallet af celler for hver unikke arealanvendelsesklasse, hvorefter det påvirkede areal for hver arealklasse beregnes som andel af det samlede oversvømmede areal og visualiseres som et søjlediagram. 

\subsection{Fremskrivning af 2023 stormfloden og en statistisk 100-års hændelse} \label{Afsnit: Fremskrivning og statistisk}

Til at undersøge hvordan studieområderne vil blive påvirket i lyset af fremtidige klimaforandringer især af risikoen for stigende vandstand. Til dette er der blevet undersøgt to forskellige metoder til at undersøge påvirkningen af klimaforandringer: et statistisk oversvømmelsesniveau ved en 100-års hændelse som præsenteret af Klimaatlas fra \cite{dmi_data_2025} og en fremskrivning af oktober 2023 stormfloden til slutningen af det nuværende århundrede baseret på projektioner fra den sjette \cite{ipcc_report_AR6} rapport.\\

{\large \textit{Statistisk 100-års hændelse}}\\
For at undersøge påvirkningen af studieområderne baseret på en statistisk 100-års hændelse, blev DMIs Klimaatlas anvendt til at finde vandstandshøjden af en stormflod der statistisk set vil optræde en gang per 100 år. Klimaatlas inddeler Danmarks kyster i kyststrækninger og de fire kyststrækninger der blev anvendt er: Lillebælt Syd for Aabenraa, Femern Bælt for Gedser, Falsters og Møns Østersøkyst for Hesnæs og Faxe Bugt for Præstø. For hver kyststrækning bruges den absolute vandstandshøjde for en statistisk 100-års hændelse ved to Shared Socioeconomic Pathways (SSP) scenarie: et mellemhøjt udledningsscenarie (SSP2,5-4,5) og et meget højt udledningsscenarie (SSP5-8,5). \\
Herefter bruges Inundation Modellen til at simulere op til SSP5-8,5 scenariet for hvert studieområde. Processen bag dette er den samme som i afsnit \ref{Afsnit: Simulering af stormflod 2023}, men i stedet for at starte ved 100 cm starter modellen ved den målte oktober 2023 vandstandshøjde for hvert studieområde. Der blev simuleret op til en vandstandshøjde på 251 cm for Aabenraa, 242 cm for Gedser, 253 cm for Hesnæs og 225 cm for Præstø. \\

{\large \textit{Fremskrivning}} \\
Til at fremskrive hvordan oktober 2023 stormfloden vil se ud i slutningen af det nuværende århundrede på baggrund af stigende havspejl blev der lavet en udregning af middelvandstanden i slutningen af århundredet. DMIs Klimaatlas har allerede tal for middelvandstanden i slutningen af århundredet, men de er i forhold til en referenceperiode fra 1981-2010. Det har derfor været nødvendigt at finde middelvandstandsniveauet i 2023 i forhold til DMIs referenceperiode. \\

Til udregningen af dette er der blev der brugt et værktøj udarbejdet af \cite{NASA_tool} til at få havspejlsstignings projektioner fra \cite{garner_ipcc_2021}. Værktøjet bruger en referenceperiode fra 1995-2014. I værktøjet vælges der den nærmeste Permanent Service for Mean Sea Level (PSMSL) station for hvert studieområde. PSMSL stationen blev udvalgt ud fra det hav basin der var tættest på hvert studieområde. Dette var Fynshav for Aabenraa, Gedser Havn for Gedser, Warnemünde for Hesnæs og Skanör for Præstø. \\

Værktøjet giver derefter en tabel med alle de individuelle bidrag til havspejlsstigning og en median havspejlsstigning ($SLR_{50}$) for hver station. For at udregne middelvandstanden i 2023, er projektionen for 2030 anvendt, da det er den tætteste. Til udregningen er der blevet opstillet en følgende ligning: 
\begin{align} \label{Equation: Vandstandsstigning calculation}
    S_r = S_p- \left( \frac{\Delta{t}}{\Delta{t_r}}\times SLR_{50} + \Delta{M_r} \times \left(\frac{SLR_{50}}{\Delta{t_r}}\right) \right)
\end{align}
Herefter beregnes middelvandstanden i 2023 ved at finde forholdet af differencen mellem målåret ($\Delta{t}$), NASAs referenceår ($\Delta{t_r}$) og projektionsåret. Dette forhold ganges med $SLR_{50}$ og giver middelvandstanden i 2023 i forhold til NASAs referenceperiode 1995-2014. Herefter beregnes den forventede middelvandstandsstigning fra DMIs referenceperiode (1981-2010) til NASAs referenceperiode (1995-2014). Dette er blevet gjort under den antagelse at stigningen fra DMIs referenceperiode til NASA har været lineær \citep{danish_meteorological_institute_dmi_2024}. Dette gøres ved at tage differencen mellem referenceperiodernes median ($\Delta{M_r}$) og gange med raten af havspejlstigning frem mod 2030 ($\frac{SLR_{50}}{\Delta{t_r}}$). \\
Dette giver et resultat hvad vandstanden i 2023 har været i forhold til DMIs referenceperiode og gør det muligt at fratrække den vandstand fra den projekteret vandstand i slutningen af det nuværende århundrede ($S_p$) og giver $S_r$ som er middelvandstanden i slutningen af århundredet med 2023 som det nye referenceår. Denne fremgangsmåde blev udført ved et SSP2-4,5 og et SSP5-8,5 scenarie.\\
Dette gør det dermed muligt at fremskrive hvordan oktober 2023 stormfloden vil se ud hvis den skete i slutningen af århundredet.

Herefter blev Inundation Modellen igen kørt for at få nye oversvømmelsesrastere. Der blev simuleret op til 268 cm for Aabenraa, 259 for Hesnæs og 246 for Præstø. Der blev ikke simuleret nye oversvømmelsesrastere for Gedser da den statistiske 100-års hændelse ved SSP8.5 er en højere vandstand end den fremskrevet 2023 stormflod til slutningen af århundredet. \\


Derefter blev alle oversvømmelsesrasterne konverteret til polygoner og kombineret med værktøjet \textit{"Merge"} for at få visualiseret forskellene mellem en statistisk 100-års hændelse og en fremskrivning af oktober 2023 stormfloden på et samlet kort der viser hvordan en statistisk 100-års hændelse og en fremskrivning af oktober 2023 stormfloden vil påvirke studieområderne.


\newpage
\section{Resultater}

På baggrund af simuleringer med Inundation Modellen og analysering af resultaterne præsenteres resultaterne af 2023-stormflods simuleringen, påvirkede arealanvendelser og simuleringerne af fremtidige stormfloder ved en statistisk 100-års hændelse og en fremskrevet stormflod i et SSP4,5 og 8,5 udledningsscenarie. Resultaterne af den simulerede stormflod sammenlignes med den observerede stormflod.

\subsection{Simulerede 2023-stormflod og påvirkede arealanvendelser}
Figur \ref{Subfig: Målt Aabenraa} viser det oversvømmelseskort Aabenraa kommune har udarbejdet, der viser den observerede hændelse fra stormfloden og figur \ref{Subfig: Model Aabenraa} viser resultatet af simuleringen af stormfloden for Aabenraa. 
\begin{figure}[H]
    \begin{subfigure}[t]{0.5\textwidth}
        \centering
        \includegraphics[width=0.95\linewidth]{images/Resultater/2023Malt/2023 resultat_aabenraa.jpg}
        \caption{}
        \label{Subfig: Målt Aabenraa}
    \end{subfigure}
    \begin{subfigure}[t]{0.5\textwidth}
        \centering
        \includegraphics[width=0.95\linewidth]{images/Resultater/2023Model/2023 model_aabenraa.jpg}
        \caption{}
        \label{Subfig: Model Aabenraa}
    \end{subfigure}
    \caption{Oversvømmelseskort over 2023-stormfloden for Aabenraa. \textbf{(a)} Observeret data \textbf{(b)} Simuleret data.}
    \label{Figur: Målt & simuleret Aabenraa}
\end{figure}
Der er forskel mellem de to resultater. Modellens resultat er større, især vest for lystbådehavnen ved et blandet rekreativt og bebygget område. Modellen resulterede i et areal på 129,8 ha, mens det observeret areal var på 70,1 ha. Dette medfører i at modellens resultat er 83,5\% større end den observeret hændelse, svarende til ca. 59 ha.\\

Stormfloden i Aabenraa påvirkede syv forskellige arealanvendelser. I figur \ref{Subfig: Arealklasser i procent Aabenraa} er der vist hvilke arealer der blev påvirket under den observeret og simuleret stormflod, som procent af det totale oversvømmet areal. Under den observeret stormflod blev 42\% af bebyggede områder oversvømmet, svarende til 30 ha (figur \ref{Subfig: Hektar arealklasser Aabenraa}). Hvorimod ved den simuleret stormflod blev 29\% af det bebyggede areal oversvømmet, svarende til ca. 37 ha (figur \ref{Subfig: Hektar arealklasser Aabenraa}). En anden klasse hvor der er forskel mellem resultaterne er rekreativt areal, hvor 25 ha oversvømmes i den simuleret stormflod, svarende til ca. 20\% af det totale areal. I modsætning til den observeret stormflod hvor 0,41 ha (0,6\%) af rekreativ areal blev oversvømmet.
\begin{figure}[H]
    \begin{subfigure}[t]{0.5\textwidth}
        \centering
        \includegraphics[width=1\linewidth]{images/Resultater/areal_anvendelses_grafer/aabenraa_arealanvendelse.jpg}
        \caption{}
        \label{Subfig: Arealklasser i procent Aabenraa}
    \end{subfigure}
    \begin{subfigure}[t]{0.5\textwidth}
        \centering
        \includegraphics[width=1\linewidth]{images/Resultater/areal_anvendelses_grafer/aabenraa_oversvommet_Hektar.jpg}
        \caption{}
        \label{Subfig: Hektar arealklasser Aabenraa}
    \end{subfigure}
    \caption{Påvirkede arealanvendelsesklasser i Aabenraa for den observeret og simuleret stormflod. \textbf{(a)} Oversømmet areal som procent af det totale areal. \textbf{(b)} Oversvømmet areal i hektar.}
    \label{Figur: Påvirket arealanvendelse Aabenraa}
\end{figure}
  
\begin{figure}[H]
    \begin{subfigure}[t]{0.5\textwidth}
        \centering
        \includegraphics[width=0.95\linewidth]{images/Resultater/2023Malt/2023 resultat_gedser.jpg}
        \caption{}
        \label{Subfig: Målt Gedser}
    \end{subfigure}
    \begin{subfigure}[t]{0.5\textwidth}
        \centering
        \includegraphics[width=0.95\linewidth]{images/Resultater/2023Model/2023 model_gedser.jpg}
        \caption{}
        \label{Subfig: Model Gedser}
    \end{subfigure}
    \caption{Oversvømmelseskort over 2023-stormfloden for Gedser Havn. \textbf{(a)} Observeret data. \textbf{(b)} Simuleret data}
    \label{Figur: Målt & simuleret Gedser}
\end{figure}
På figur \ref{Subfig: Målt Gedser} ses arealet af den observeret oversvømmelse af Gedser Havn af stormfloden. På figur \ref{Subfig: Model Gedser} er det simulerede resultat. Resultatet fra Inundation Modellen er ens med den observeret hændelse. Den observeret hændelse havde et oversvømmet areal på 34,5 ha kontra 33,2 ha fra Inundation Modellen. Modellens resultat er dermed ca. 4\% mindre, svarende til ca. 1,3 ha. Det er primært arealet omkring lystbådehavnen nord for færgehavnen og et lavliggende naturområde vest for færgehavnen der blev oversvømmet. Dette fremgår af både det observeret data og det simuleret resultat.\\

I Gedser Havn blev otte arealklasser påvirket af oversvømmelserne under stormfloden. Ved både den observeret og simuleret stormflod er det naturområder der blev mest påvirket med 17,7 og 17,2 ha, svarende til henholdsvis 51,5 og 51,7\% af det samlet areal (figur \ref{Subfig: Procent areal Gedser} og \ref{Subfig: Hektar areal Gedser}).
Andelen og arealet af de andre oversvømmet arealer er ens for begge resultater (figur \ref{Figur: Påvirket arealanvendelse Gedser}).
\begin{figure}[H]
    \begin{subfigure}[b]{0.5\textwidth}
        \centering
        \includegraphics[width=1\linewidth]{images/Resultater/areal_anvendelses_grafer/gedser_arealanvendelse.jpg}
        \caption{}
        \label{Subfig: Procent areal Gedser}
    \end{subfigure}
    \begin{subfigure}[b]{0.5\textwidth}
        \centering
        \includegraphics[width=1\linewidth]{images/Resultater/areal_anvendelses_grafer/gedser_oversvommet_Hektar.jpg}
        \caption{}
        \label{Subfig: Hektar areal Gedser}
    \end{subfigure}
    \caption{Påvirkede arealanvendelsesklasser i Gedser Havn for den observeret og simuleret stormflod. Bemærk at kategorien "Erhverv" \hspace{0.1cm}udgør >0,05\% og er derfor ikke synlig. \textbf{(a)} Oversømmet areal som procent af det totale areal. \textbf{(b)} Oversvømmet areal i hektar.}
    \label{Figur: Påvirket arealanvendelse Gedser}
\end{figure}



\begin{figure}[H]
    \begin{subfigure}[t]{0.5\textwidth}
        \centering
        \includegraphics[width=0.95\linewidth]{images/Resultater/2023Malt/2023 resultat_hesnaes.jpg}
        \caption{}
        \label{Subfig: Målt Hesnæs}
    \end{subfigure}
    \begin{subfigure}[t]{0.5\textwidth}
        \centering
        \includegraphics[width=0.95\linewidth]{images/Resultater/2023Model/2023 model_hesnaes.jpg}
        \caption{}
        \label{Subfig: Model Hesnæs}
    \end{subfigure}
    \caption{Oversvømmelseskort over 2023-stormfloden for Hesnæs. \textbf{(a)} Observeret data. \textbf{(b)} Simuleret data.}
    \label{Figur: Målt & simuleret Hesnæs}
\end{figure} 
For Hesnæs var det primært havnen, inklusiv havnens mole og kystlinjen der blev oversvømmet under stormfloden (figur \ref{Subfig: Målt Hesnæs}). Resultatet fra Inundation Modellen i figur \ref{Subfig: Model Hesnæs} giver det samme resultat og der er er minimal forskel mellem det observeret og simuleret resultat. Den observeret hændelse havde et totalt oversvømmet areal på 3,5 ha og det simulerede resultat havde et areal på 3,3 ha. Modellens resultat er derfor 0,2 ha mindre, svarende til en forskel på 5,2\%.\\

I Hesnæs blev seks forskellige arealanvendelser påvirket. De mest påvirkede arealer i Hesnæs var kysten, der falder ind under naturområder. Dette har været gældende for begge resultater. For den observeret hændelse blev 66,2\% naturområde oversvømmet og for den simuleret hændelse blev 65,8\% oversvømmet (figur \ref{Subfig: Procent hesnæs}). Dette svarer til henholdsvis 2,3 og 2,2 ha (figur \ref{Subfig: Hektar Hesnæs}). For de andre arealklasser er resultaterne ens for den observeret og den simuleret stormflod. 
\begin{figure}[H]
    \begin{subfigure}[b]{0.5\textwidth}
        \centering
        \includegraphics[width=1\linewidth]{images/Resultater/areal_anvendelses_grafer/hesnaes_arealanvendelse.jpg}
        \caption{}
        \label{Subfig: Procent hesnæs}
    \end{subfigure}
    \begin{subfigure}[b]{0.5\textwidth}
        \centering
        \includegraphics[width=1\linewidth]{images/Resultater/areal_anvendelses_grafer/hesnaes_oversvommet_Hektar.jpg}
        \caption{}
        \label{Subfig: Hektar Hesnæs}
    \end{subfigure}
    \caption{Påvirkede arealanvendelsesklasser i Hesnæs for den observeret og simuleret stormflod. \textbf{(a)} Oversømmet areal som procent af det totale areal. \textbf{(b)} Oversvømmet areal i hektar.}
    \label{Figur: Påvirket arealanvendelse Hesnæs}
\end{figure}



\begin{figure}[H]
    \begin{subfigure}[t]{0.5\textwidth}
        \centering
        \includegraphics[width=0.95\linewidth]{images/Resultater/2023Malt/2023 resultat_praestoe.jpg}
        \caption{}
        \label{Subfig: Målt Præstø}
    \end{subfigure}
    \begin{subfigure}[t]{0.5\textwidth}
        \centering
        \includegraphics[width=0.95\linewidth]{images/Resultater/2023Model/2023 model_praestoe.jpg}
        \caption{}
        \label{Subfig: Model Præstø}
    \end{subfigure}
    \caption{Oversvømmelseskort over oktober 2023 stormfloden for Præstø. \textbf{(a)} Observeret data. \textbf{(b)} Simuleret data}
    \label{Figur: Målt & simuleret Præstø}
\end{figure}
Figur \ref{Figur: Målt & simuleret Præstø} viser den observeret stormflods hændelse og den simuleret hændelse i Præstø. Begge kort viser at kysten og havnen af Præstø bliver oversvømmet samt den nordlige bydel. Forskellen mellem de to oversvømmelseskort er at under den observeret hændelse blev en del af Tubæk Ådal samt de omkringliggende arealer oversvømmet. Den observeret hændelse på 53,5 ha er derfor et 26\% større areal end det simuleret resultat på 39.8 ha. Denne forskel svarer til 13,7 ha.\\


\begin{figure}[H]
    \begin{subfigure}[b]{0.5\textwidth}
        \centering
        \includegraphics[width=1\linewidth]{images/Resultater/areal_anvendelses_grafer/praestoe_arealanvendelse.jpg}
        \caption{}
        \label{Subfig: Procent præstø}
    \end{subfigure}
    \begin{subfigure}[b]{0.5\textwidth}
        \centering
        \includegraphics[width=1\linewidth]{images/Resultater/areal_anvendelses_grafer/praestoe_oversvommet_Hektar.jpg}
        \caption{}
        \label{Subfig: Hektar præstø}
    \end{subfigure}
    \caption{Påvirkede arealanvendelsesklasser i Præstø for den observeret og simuleret stormflod. \textbf{(a)} Oversømmet areal som procent af det totale areal. \textbf{(b)} Oversvømmet areal i hektar.}
    \label{Figur: Påvirket arealanvendelse Præstø}
\end{figure}
I Præstø blev otte forskellige arealanvendelser påvirket af stormfloden. For både den observeret og simuleret hændelse var naturområde det mest påvirket areal med 47 og 60\% henholdsvis. Da den observeret oversvømmelse udgør et større samlet areal udgør naturområder en mindre andel af det totale areal end for den simuleret hændelse (figur \ref{Subfig: Procent præstø}). Den observeret hændelse havde et større rekreativt område oversvømmet end den simuleret hændelse med en forskel på 4 ha, da arealerne langs Tubæk Ådal er klassificeret som rekreative områder. Grundet oversvømmelsen af Tubæk Ådal er et større areal bebyggede områder blevet oversvømmet med ca. 12 ha i den observeret stormflod og 6 ha i den simuleret stormflod. Det samme er gældende for infrastruktur og erhverv (figur \ref{Subfig: Hektar præstø}).


\newpage
\subsection{Fremskrevet og 100-års stormflodshændelser}

I dette afsnit præsenteres resultaterne af en simuleret 100-års hændelse og fremskrevet 2023-stormfloden i 2100 ved et SSP4,5 og 8,5 udledningsscenarie. 
\begin{figure}[H]
    \centering
    \includegraphics[width=0.8\linewidth]{images/Resultater/fremskrevet_hændelser/klima resultat_aabenraa.jpg}
    \caption{Kort over oversvømmet areal i Aabenraa af en stormflod ved en statistisk 100-års hændelse og en fremskrivning af 2023-stormfloden til slutningen af århundredet ved et SSP4,5 og SSP8,5 scenarie.}
    \label{Figur: Klima Aabenraa}
\end{figure}
I figur \ref{Figur: Klima Aabenraa} ses et oversvømmelseskort over Aabenraa for en 100-års hændelse og en fremskrivning af stormfloden ved et SSP4,5 og SSP8,5 scenarie. Det største areal oversvømmet sker ved en 100-års hændelse i SSP4,5 med en vandstand på 234 cm over DVR90 hvor en større del af den sydlige bydel bliver oversvømmet. Det oversvømmet areal ved fremskrivningen af stormfloden i et SSP8,5 scenarie er 13\% større end en 100-års hændelse i SSP4,5. Dette svarer til 19,6 ha. I forhold til den observeret 2023-stormflodshændelse, oversvømmer en fremskrevet stormflod et 96 og 104 ha større areal ved henholdsvis et SSP4,5 og SSP8,5 klimascenarie. Det er primært i den nordelige del af Aabenraa hvor der er forskel mellem 100-års hændelsen og den fremskrevet stormflod.  
\begin{figure}[H]
    \centering
    \includegraphics[width=0.8\linewidth]{images/Resultater/fremskrevet_hændelser/klima resultat gedser.jpg}
    \caption{Kort over oversvømmet areal i Gedser Havn af en stormflod ved en statistisk 100-års hændelse og en fremskrivning af oktober 2023 stormfloden til slutningen af århundredet ved et SSP4,5 og SSP8,5 scenarie.}
    \label{Figur: Klima Gedser Havn}
\end{figure}
I figur \ref{Figur: Klima Gedser Havn} er der vist et oversvømmelseskort over Gedser Havn for en 100-års hændelse og fremskrivningen af stormfloden ved et SSP4,5 og 8,5 scenarie. Den største del af oversvømmelsen sker ved en fremskrevet stormflod ved SSP4,5 med en vandstand på 220 cm over DVR90. Størstedelen af landsbyen Gedser Havn bliver oversvømmet samt en del af det bagvedliggende landbrugsområder. Ved en 100-års hændelse i et SSP8,5 scenarie er det oversvømmet areal steget til 219,7 ha, en stigning på 537\% i forhold til den stormflod der ramte Gedser i 2023. Den inderste del af færgehavnen og terminalen bliver ikke oversvømmet ved nogen af de fire scenarier, dog bliver hovedvejen ind til færgehavnen oversvømmet.
\begin{figure}[H]
    \centering
    \includegraphics[width=0.8\linewidth]{images/Resultater/fremskrevet_hændelser/klima resultat hesnaes.jpg}
    \caption{Kort over oversvømmet areal i Hesnæs af en stormflod ved en statistisk 100-års hændelse og en fremskrivning af oktober 2023 stormfloden til slutningen af århundredet ved et SSP4,5 og SSP8,5 scenarie.}
    \label{Figur: Klima Hesnæs}
\end{figure}
I Hesnæs bliver et større lavliggende blandet græs- og landbrugsareal oversvømmet ved en 100-års hændelse i SSP8,5 scenariet. Ved en 100-års hændelse i SSP4,5 oversvømmes det meste af kysten og havnen. Selve Hesnæs by er udenfor oversvømmelsesrisiko selv ved det højeste vandstandsniveau på 259 cm DVR90 (figur \ref{Figur: Klima Hesnæs}). Ved en fremskrevet stormflod i SSP8,5, vil det oversvømmede areal stige fra 4,8 ha til 114,8 ha. Dette er en stigning på 2273\%. \\
Ved en fremskrevet 2023 stormflod ved SSP8,5 vil der oversvømmes et areal der er 3219\% større end det der blev målt under stormfloden i 2023. Dette er en stigning på 111 ha. \\

I figur \ref{Figur: Klima Præstø} ses vandets udbredelse ved en 100-års hændelse og en fremskrevet stormflod i et SSP4,5 og 8,5 scenarie for Præstø. Ved en 100-års SSP4,5 hændelse bliver dele af Præstø by oversvømmet, både nord og syd for Tubæk Ådal. Størstedelen af Tubæk Ådal vil også blive lagt under vand og åen vil gå over sine bredder i den sydlige del af byen. Ved vandstande på over 225 cm vil vandet følge Tubæk Å længere ind over land. Et areal der er 68\% større end stormfloden der oversvømmede byen i 2023 vil blive oversvømmet hvis den samme stormflod rammer Præstø i slutningen af dette århundrede i et SSP8,5 scenarie. Dette svarer til et 36 ha større oversvømmet areal.
\begin{figure}[H]
    \centering
    \includegraphics[width=0.8\linewidth]{images/Resultater/fremskrevet_hændelser/klima resultat praestoe.jpg}
    \caption{Kort over oversvømmet areal i Præstø af en stormflod ved en statistisk 100-års hændelse og en fremskrivning af oktober 2023 stormfloden til slutningen af århundredet ved et SSP4,5 og SSP8,5 scenarie.}
    \label{Figur: Klima Præstø}
\end{figure}
Sammenfattet bliver alle studieområderne påvirket af oversvømmelser ved stormfloder. I tabel \ref{Tabel: Oversvømmet arealer af stormfloder} er der blevet samlet størrelsen af oversvømmelserne i hektar for alle hændelser simuleret med Inundation Modellen samt den observeret stormflod i 2023. Det største oversvømmet areal er i Gedser Havn ved en fremskrevet stormflod i SSP8,5 på 219,7 ha og den mindste areal er simuleringen af 2023 stormfloden i Hesnæs på 3,3 ha. Gedser Havn er det område med den højeste absolutte stigning i oversvømmet areal med 183,6 ha fra den observerede stormflod til den fremskrevne stormflod i SSP8,5. 
\begin{table}[H]
\centering
\renewcommand{\arraystretch}{1.2} 
\begin{threeparttable}
\caption{Oversvømmet areal af den observerede stormflod, den simuleret stormflod samt den statistiske 100-års hændelse og den fremskrevet stormflod til slutningen af århundredet ved SSP4.5 og 8.5 i hektar for hvert studieområde.}
    \begin{tabular}{@{} l S[table-format=3.1, output-decimal-marker={,}] 
                        S[table-format=3.1, output-decimal-marker={,}] 
                        S[table-format=3.1, output-decimal-marker={,}] 
                        S[table-format=3.1, output-decimal-marker={,}] 
                        S[table-format=3.1, output-decimal-marker={,}] 
                        S[table-format=3.1, output-decimal-marker={,}] @{}} 
    \toprule
    & \multicolumn{1}{c}{\textbf{\makecell{Observeret 2023\\stormflod}}} 
    & \multicolumn{1}{c}{\textbf{\makecell{Simuleret 2023\\stormflod}}} 
    & \multicolumn{2}{c}{\textbf{\makecell{Statistisk 100-års\\hændelse}}} 
    & \multicolumn{2}{c}{\textbf{\makecell{Fremskrevet 2023\\stormflod}}} \\ 
    \cmidrule(l){4-5} \cmidrule(l){6-7}
    {\textit{ha}} & & & {\textit{SSP4.5}} & {\textit{SSP8.5}} & {\textit{SSP4.5}} & {\textit{SSP8.5}} \\
    \midrule
      Aabenraa & 70,7 & 129,8 & 155,2 & 166,7 & 166,7 & 174,9 \\
      Gedser & 34,5 & 33,2 & 205,1 & 219,7 & 201 & 218,1 \\ 
      Hesnæs & 3.5 & 3,3 & 4,8 & 113,4 & 4,8 & 114,8 \\
      Præstø & 53,5 & 39,8 & 75,7 & 82 & 84,1 & 89,7 \\
    \bottomrule
    \end{tabular}
\label{Tabel: Oversvømmet arealer af stormfloder}
\end{threeparttable}
\end{table}


%\begin{table}[H]
%\centering
%\begin{threeparttable}
%\caption{Oversvømmet areal af den observeret 2023-stormflod, den simuleret 2023-stormflod samt den statistiske 100-års hændelse og den fremskrevet 2023-stormflod til slutningen af århundredet ved SSP4.5 og 8.5 i hektar og den procentvise forskel i forhold til 2023 stormfloden for hvert studieområde.}
%\begin{tabular}{@{} l >{\centering\arraybackslash}m{0.13\textwidth} >{\centering\arraybackslash}m{0.13\textwidth} >{\centering\arraybackslash}m{0.13\textwidth} >{\centering\arraybackslash}m{0.13\textwidth} >{\centering\arraybackslash}m{0.13\textwidth} >{\centering\arraybackslash}m{0.13\textwidth} @{}} 
%\toprule
%& \textbf{\makecell{Målt 2023\\stormflod}} 
%& \textbf{\makecell{Simuleret 2023\\stormflod}} 
%& \multicolumn{2}{c}{\textbf{\makecell{Statistisk 100-års\\hændelse}}} 
%& \multicolumn{2}{c}{\textbf{\makecell{Fremskrevet 2023\\stormflod}}} \\ 
%\cmidrule(l){4-5} \cmidrule(l){6-7}
%{\textit{ha}} & & & {\textit{SSP4.5}} & {\textit{SSP8.5}} & {\textit{SSP4.5}} & {\textit{SSP8.5}} \\
%\midrule
%\renewcommand{\arraystretch}{1.2}
%Aabenraa & 70,7 & 129,8 (+84\%) & 155,2 (+119\%) & 166,7 (+136\%) & 166,7 (+136\%) & 174,9 (+147\%)\\
%Gedser   & 34,5 & 33,2 (-4\%) & 205,1 (+495\%)& 219,7 (+537\%)& 201,0 (+483\%)& 218,1 (+533\%)\\ 
%Hesnæs   & 3,5  & 3,3 (-5\%) & 4,8 (+38\%)  & 113,4 (+3181\%) & 4,8 (+40\%) & 114,8 (+3220\%)\\
%Præstø   & 53,5 & 39,8 (-26\%)& 75,7 (+42\%)& 82,0 (+53\%)& 84,1 (+57\%)& 89,7 (+68\%)\\
%\bottomrule
%\end{tabular}
%\label{Tabel: Oversvømmet arealer af stormfloder i hektar}
%\end{threeparttable}
%\end{table}
 

\section{Diskussion}



\subsection{Resultat diskussion}

På baggrund af resultaterne er der en række elementer der kan diskuteres. Resultaterne fra Inundation Modellen til at simulere 2023-stormfloden har været fornuftigt. To af studieområderne, Gedser Havn og Hesnæs, har haft simulerings resultater der er tæt afspejler det observerede data (figur \ref{Figur: Målt & simuleret Gedser} og \ref{Figur: Målt & simuleret Hesnæs}). For Geder Havn og Hesnæs har Inundation modellen været mindre i udstræk end det observeret data. \\
Aabenraa og Præstø derimod har haft henholdsvis et større og mindre areal oversvømmet i Inundation Modellens resultat end det der skete under stormfloden grundet en række eksterne faktorer der skete under selve stormfloden.\\

For Aabenraa er forskellen mellem det observeret data og simuleringen forårsaget af en række beredskabs tiltag der blev implementeret langs havnen i løbet af den 20. oktober 2023 (figur \ref{Figur: Beredskabstiltag}). Dette har inkluderet en ca. 1 km lang watertube og barrikader af BigBag sandsække langs vejen ved havnen. Det sorte kryds i figur \ref{Figur: Beredskabstiltag} indikerer hvor watertuben bristede på grund af vandpresset. På trods af dette brist formåede det lokale beredskab Brand og Redning Sønderjylland at få lavet en dæmning af sække med jord der hindrede vandets strømning mod syd og længere ind i Aabenraa midtby og dermed det bagvedliggende område der blev oversvømmet i Inundation Modellens resultat i figur \ref{Subfig: Model Aabenraa}. \\
\begin{figure}[H]
    \centering
    \includegraphics[width=0.8\linewidth]{images/diskussion/beredskabstiltag.jpg}
    \caption{Kort over beredskabstiltag foretaget i løbet af den 20. oktober 2023 i Aabenraa og den observeret oversvømmelse. Dæmningen af jord blev etableret efter watertuben bristede. Kilde: Aabenraa Kommune}
    \label{Figur: Beredskabstiltag}
\end{figure}
Udover dette har der været to områder i det observeret data er der blev konstateret oversvømmelse, men hvor modellen ikke indikerede en oversvømmelse. De to områder er to mindre vandløb mellem Aabenraa by og Ensted Industrihavn og noget af det bagvedliggende areal (figur \ref{Figur: Tilpasnings fejl}). Denne forskel skyldes sandsynligvis en datafejl i kortet fra Aabenraa Kommune. 
\begin{figure}[H]
    \centering
    \includegraphics[width=0.8\linewidth]{images/diskussion/tilpasnings_error.jpg}
    \caption{Kort over to hydrologiske tilpasninger der ved fejl ikke er implementeret korrekt i kortet modtaget fra Aabenraa Kommune.}
    \label{Figur: Tilpasnings fejl}
\end{figure}
Begge vandløb har en udmunding ud til fjorden. Den nordligste tilpasning er en hesteskotilpasning, med en klassifikation fra GeoDanmark som en højvandssluse. Denne ville have været lukket under stormfloden, og der er ingen observationer om at denne sluse brød sammen under stormfloden. Den anden tilpasning længere sydpå er en kontraklap til at føre vand væk fra landet til havet under et skybrud. Begge objekter er aktuelle og aktive fra 2015, så antagelsen om ændret brug må forkastes. Det er derfor mere sandsynligt at de to hydrologiske tilpasninger er blevet glemt at inkludere i datasættet, da andre lignende tilpasninger er blevet korrekt implementeret eller at vandet på anden vis er trængt ind over. \\  


I Præstø har den primære forskel mellem det observeret data og simuleringen været forårsaget af sammenbruddet af sluseporten ved udmundingen af Tubæk Å. Dette resulterede i at Tubæk Ådal og dele af byen blev oversvømmet. Dette er ikke blevet fanget af Inundation Modellen, da det er en uforudset hændelse. Det kan derfor diskuteres hvorvidt den observeret oversvømmelse og modellen ville være mere ens hvis sluseporten ikke brød sammen. \\
Samtidig kan det diskuteres om der blev lavet barrikader længere opstrøms af Tubæk Å, da oversvømmelseskortet fra Vordingborg Kommune stopper abrupt når åen passerer under en vej syd for byen. Der er ikke noget der indikerer at vandet er blevet bremset under denne vej og det må derfor at antages at kommunen ikke har lavet undersøgelser af oversvømmelsesniveauet forbi dette punkt. \\

% arealanvendelse 
Til kvantificeringen af påvirkede arealanvendelser blev der foretaget en reklassifikation af arealklasserne præsenteret i BaseMap fra \cite{Jepsen_levin_2013, levin_basemap04_2022}. Det primære formål med reklassifikationen var at reducere antallet af arealklasser for overskuelighedens skyld samt en filtrering af nødvendige klasser såsom søer, vandløb, hav osv. Reduktionen af arealklasser har derfor betydet at kvantificeringen af påvirkede arealer er blevet mindre transparent og mere generaliseret i det endelige resultat. I tilfældet med Aabenraa og Præstø hvor det simulerede resultat varierede fra det observerede data, ville de originale 35 klasser kunne beskrive mere detaljerede forskelle i påvirkede arealanvendelser end de otte overordnede klasser. \\
Desuden blev der udregnet en procentvis andel af det totale oversvømmede areal til at visualisere hvor meget oversvømmet areal hver klasse udgjorde det samlede areal. Dette har været misvisende som set i figur \ref{Figur: Påvirket arealanvendelse Præstø} hvor naturområder i Præstø for det simuleret resultat har en højere andel oversvømmet af det totale areal, men det svarer til et mindre areal i hektar end det observeret data. \\ 


Simuleringerne af fremtidige stormfloder i form af statistiske 100-års hændelser og fremskrivningen af 2023-stormfloden har været fornuftige. Da resultatet af 100-års hændelsen er direkte fra DMI er resultatet her det forventede. Fremskrivningen af stormfloden er derimod det mere interessante. Fremskrivningen blev baseret på medianværdien af IPCCs projektioner for 2030 samt DMIs forventede vandstandsniveau ved udgangen af det 21. århundrede. Formålet var at bestemme middelvandstanden i 2023 i forhold til DMIs reference i 1995. Denne metode indebar en vis usikkerhed, idet den forudsætter, at havspejlsstigningen siden 1995 har været lineær. Her er det værd at bemærke at denne usikkerhed ikke umiddelbart kan dokumenteres, da \cite{danish_meteorological_institute_dmi_2024} ligeledes antager den samme lineær udvikling. Det er derfor væsentligt at understrege at denne udregning forudsætter at stigningen er lineær og det derfor ikke nødvendigvis afspejler observeret data og ligeledes ikke er tilfældet alle steder i verden. \\
Uanset denne antagelse har fremskrivningen af stormfloden resultateret i fornuftige værdier, der ikke er urealistiske for lokaliteten og for det SSP scenarie de blev beregnet på.\\


\subsection{Inkluderingen af bygninger i DHyM}
En af de prominente metodiske overvejelser, der blev taget relativt tidligt i arbejdsprocessen var at inkludere bygninger i den digitale hydrologiske model (DHyM), som ikke passerbare enheder i terrænet. Denne beslutning blev truffet ved at argumentere for at vand ikke kan passere direkte igennem bygninger, når en oversvømmelse sker og for at sætte et ens datagrundlag for at kunne sammenligne Inundation Modellens resultat med det modtaget data fra kommunerne. Dertil er det ikke en normal procedure at tilføje bygninger til en DHyM \citep{khosh_bin_ghomash_technical_2024}, da der ofte bruges en DTM til hydrologisk modellering fremfor en digital overflade model (DSM), som inkluderer bygninger og træer. Derudover fjernes muligheden for potentielle undersøgelser og analyser af stormfloders påvirkning på bestemte bygninger og bygningstyper samt vurderinger af bygningers risici ved fremtidige oversvømmelser ved at tilføje bygninger til DHyM. Dette har været tilfældet ved denne undersøgelse. \\
Derudover blev alle bygninger tildelt den samme højdeværdi på 20 meter. En mere nøjagtig fremgangsmåde ville have været at udregne den maksimale taghøjde af bygningerne for at sikre korrekt repræsentation af virkeligheden, men da denne undersøgelse primært har fokuseret på at vandet ikke skulle kunne passere igennem en bygning har dette ikke været en prioritet. \\


\subsection{Begrænsninger ved Inundation Modellen}
Inundation Modellen har været et godt redskab til simulere oversvømmelsesudbredelsen ved bestemte vandstandsniveauer og været fornuftigt til at simulere den stormflods hændelse der skete i oktober 2023. 
Den største fordel modellen har, er dens brugervenlighed og dens relative simple struktur. Dette gør det nemt for individuelle personer, studerende og organisationer at anvende modellen. Dertil er det en fordel at modellen opererer i et GIS-regi og kan dermed kombineres med andre geospatiale data, såsom arealanvendelsesdata i dette projekt, men også andet punkt-, linje- og polygondatasæt for at undersøge risici ved oversvømmelser på sektorer og samfundskritisk infrastruktur. Modellen er også relativ hurtig i simuleringstid, hvis den hardware den udføres på er god. Simuleringstiden for resultaterne i dette projekt har været samlet 16 timer og 45 minutter og har derfor potentielt været en del længere end hvis modellen blev udført på kraftigere hardware. Simuleringstiden må forventes at kunne mere end halveres hvis hardwaren er kraftigere.\\

Derudover er der flere værktøjer i Inundation Modellens pakke, som ikke er blevet anvendt i dette projekt, herunder et værktøj der kan bestemme hvor vandet trængte ind og dermed identificere sårbare lokaliteter, der kan bruges i kommunernes planlægning af kyst- og stormflodssikring. I kombination med dette værktøj kan Inundation Modellen igen køres for aedt tjekke hvorvidt implementeringen af stormflodssikring har haft en effekt. Koblet med at modellen også er relativ hurtig i simuleringstid er dette en arbejdstilgang der kan være brugbar for både kommuner og private.\\


På trods af dette er der derimod også en række begrænsninger ved modellen. Modellen er en 100\% statisk numerisk model og det betyder derfor at der ikke er mulighed for at inkorporere en tidslig faktor. Mange stormfloder sker over en periode på ca. en til to døgn og det betyder dermed at vandmasserne ikke oversvømmer hele arealet på ingen tid, som modellen indikerer. Denne manglende tidsskala gør også at modellen ikke tager højde for korrekt hydrodynamisk adfærd, hvor det vil tage længere tid for en vandmasse at passere igennem mindre passager, såsom rør, og kortere tid igennem bredere passager, såsom under broer.\\

Andre aspekter som modellen ikke tager højde for inkluderer bl.a. effekten af vind og bølgeskvulp. Modellen leverer derfor kun et overblik over hvor oversvømmelserne sker på baggrund af vandstandshøjden. Hvis der derfor er et beskyttende dige langs en kyststrækning, der er højere end vandstandsniveauet, så vil det ikke blive oversvømmet i modellen, hvorimod i en virkelig stormflodshændelse er det muligt at bølger og vind vil presse vandet ind over diget og dermed oversvømme det bagvedliggende område. I samme bane er modellen derfor ikke i stand til at håndtere sammensatte oversvømmelser (\textit{eng.} compound flooding), hvor områder oversvømmelses fra nedbør og havet samtidig. Dette er områder hvor kraftigere og mere avanceret modeller såsom Deltares Super-Fast Inundation of Coasts (SFINCS) og DHIs MIKE modeller er i stand til at simulere.\\

Modellen kræver dertil data af høj kvalitet. Den digitale terrænmodel (DTM) skal være i høj opløsning med en lille cellestørrelse \citep{seenath_effects_2018} for at være mere virkelighedstro samt omfattende data af hydrologiske tilpasninger er også krævet for at modellen kan simulere korrekt hydrologisk bevægelser i terrænet \citep{bales_sources_2009}. Er DTM og hydrologiske tilpasninger enten utilstrækkelige eller ikke tilstede vil det have indflydelse på kvaliteten af modellens resultat og i forlængelse modellens anvendelighed. \\

Derudover er modellen lige nu eksklusiv til ArcGIS Pro softwaren. Det betyder at modellen fungerer godt i det specifikke miljø og kan derfor nemt integreres med andre dele af Esris ArcGIS miljø, såsom ArcGIS Online, Enterprise og Field Maps, til at arbejde i felten med stormflodssikring og på tværs af samarbejdspartnere. At være begrænset til Esris ArcGIS-miljø betyder derfor også at modellen er låst bag en stejl betalingsmur især for kommercielt brug og er dermed udenfor potentiel brug fra danske kommuner og andre private interessenter, der ikke har mulighed for at imødekomme den høje pris. En løsning til dette vil være at udvikle et lignende værktøj der fungerer i et gratis open-source GIS produkt såsom QGIS. \\



% hvad gør modellen godt 
% hvad gør modellen knap så godt
% hvilke forbedringer kan implementeres?
% andre begrænsninger



  

% Udregningen af fremskrevet stormflod og relaterede usikkerheder

% something something konfidensinterval maybe?



% hvor kunne den bruges
% hvilke klienter
% hvordan kunne det implementeres
% er der begrænsninger for noget af dette

\newpage
\section{Konklusion}

% Konklusionen skal indeholde:
     % Svar på problemformuleringen
     % Svar på alle underspørgsmål i problemformuleringen
     % Svar på de metodiske begrænsninger 
     % Svar på 

%Hvordan kan GIS-modellen "Inundation Model" simulere stormflodshændelsen fra oktober 2023 og forudsige udbredelsen af fremtidige oversvømmelseshændelser?

%Til at besvare problemformuleringen stilles der følgende underspørgsmål:
%Hvor præcist kan Inundation Model simulere stormfloden der blev observeret den 20.-21. oktober 2023?
   % Til at besvare dette vil resultatet fra Inundation Model blive sammenlignet med kort over vandets udbredelse fra stormfloden i Aabenraa, Gedser Havn, Hesnæs og Præstø.
 % Hvilke arealanvendelser blev påvirket af oversvømmelserne under stormfloden?
 % Hvordan vil stormfloden se ud i slutningen af det nuværende århundrede som følge af et stigende havspejl?
 % Hvordan forventes en stormflod at påvirke studieområderne ved en statistisk 100-års hændelse ved et mellem og meget højt udledningsscenarie?
 % Diskutere hvorvidt Inundation Model kan være en brugbar del af kommunernes stormflodsanalyser og %planlægning af fremtidig kystsikring

Antallet og intensiteten af stormfloder vil i fremtiden stige i takt med et stigende havspejl. Derfor er det nødvendigt at der udvikles hurtige, præcise og tilgængelige modeller der kan simulere stormfloders indvirkning på kystbyer og samfundet.\\
Inundation Modellen har simuleret stormflodshændelsen fra 2023 med fornuftige resultater. Modellen formåede at simulere vandets udbredeslse i Gedser Havn og Hesnæs i det samme omfang, som observeret under stormfloden. Aabenraa og Præstø blev derimod henholdsvis over- og underrepræsenteret grundet eksterne faktorer, som beredskabstiltag og en ødelagt sluseport.  




\newpage
\section{Referencer}
{\fontsize{11}{13} \selectfont \bibliography{references}}




\end{document}
